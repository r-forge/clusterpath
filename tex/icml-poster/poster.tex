\documentclass[final]{beamer}
\usetheme[headheight=10cm,footheight=5cm]{boxes}
\usetheme{julien}
%\usepackage{times}
\usepackage{etex}
\usepackage{amsmath,amssymb}
\usepackage[latin1]{inputenc}
\usepackage[english]{babel}
\usepackage{helvet}
\usepackage[T1]{fontenc}
\usepackage{textcomp}
\usepackage[orientation=landscape,size=a0,scale=1.4,debug]{beamerposter}   % e.g. for DIN-A0 poster
\usepackage{exscale}
\usepackage{pst-plot}
\usepackage{pstricks-add}
\usepackage{epsfig}
\usepackage{eulervm}
\usepackage{tikz}
\usepackage{array}
\usepackage{subfigure}
\usetikzlibrary{arrows,shapes,snakes,automata,backgrounds,petri}

%\usepackage{floatflt}
%\usepackage[dvips]{graphicx}
%\usepackage{graphicx}
 
%\usepackage[size=custom,width=3.0,height=120,scale=2,debug]{beamerposter}  % e.g. custom size poster
\usepackage{beamerthemejulien}

\def\x{{\mathbf x}}
\def\z{{\mathbf z}}
\def\l{{\ell}}
\def\1{{\mathbf 1}}
\def\0{{\mathbf 0}}
\def\a{{\mathbf a}}
%\def\c{{\mathbf c}}
\def\p{{\mathbf p}}
\def\v{{\mathbf v}}
\def\n{{\mathbf n}}
\def\X{{\mathbf X}}
\def\Z{{\mathbf Z}}
\def\eg{e.g.}
\def\ZZ{{\mathcal Z}}
\def\hatx{{\mathbf {\hat x}}}
\def\hatf{{ \hat f}}
\def\hatw{{\bf \hat w}}
\def\hati{{\hat{\imath}}}
\def\tildej{\tilde{\jmath}}
\def\tildel{\tilde{\ell}}
\def\hatX{{\bf \hat X}}
\def\hatb{{\hat b}}
\def\ae{\text{~~a.e.}}
\def\as{\text{~~a.s.}}
\def\hatalpha{\mbox{\boldmath${\hat \alpha}$}}
\def\tildealpha{\mbox{\boldmath${\tilde \alpha}$}}
\def\tildebeta{\mbox{\boldmath${\tilde \beta}$}}
\def\hatDelta{\mbox{\boldmath${\hat \Delta}$}}
\def\Lambdab{{\boldsymbol\Lambda}}
\def\lambdab{{\boldsymbol\lambda}}
\def\kappab{{\boldsymbol\kappa}}
\def\rhob{{\boldsymbol\rho}}
\def\taub{{\boldsymbol\tau}}

\def\xib{{\boldsymbol\xi}}
\def\xibbar{\tilde{\boldsymbol{\xi}}}
\def\nub{{\boldsymbol\nu}}
\def\Gammab{{\boldsymbol\Gamma}}
\def\Thetab{{\boldsymbol\Theta}}
\def\Sigmab{{\boldsymbol\Sigma}}
\def\Deltab{{\boldsymbol\Delta}}
\def\betab{{\boldsymbol\beta}}
\def\epsilonb{{\boldsymbol\varepsilon}}
\def\varepsilonb{{\boldsymbol\varepsilon}}
\def\alphab{{\boldsymbol\alpha}}
\def\thetab{{\boldsymbol\theta}}
\def\gammab{{\boldsymbol\gamma}}
\def\alphabtilde{{\boldsymbol\tilde{\alpha}}}
\def\hatlambdab{{\boldsymbol{\hat\lambda}}}
\def\Gammab{{\boldsymbol\Gamma}}
\def\tildeGammab{{\bf \tilde{\boldsymbol\Gamma}}}
\def\y{{\mathbf y}}
\def\FFF{\text{F}}
\def\w{{\mathbf w}}
\def\vw{{\mathbf w}}
\def\D{{\mathbf D}}
\def\mX{{\mathbf D}}
\def\Y{{\mathbf Y}}
\def\YY{{\mathcal Y}}
\def\M{{\mathbf M}}
\def\B{{\mathbf B}}
\def\Q{{\mathbf Q}}
\def\C{{\mathbf C}}
\def\W{{\mathbf W}}
\def\d{{\mathbf d}}
\def\E{{\mathbb E}}
\def\PPP{{\mathbb E}}
\def\EE{{\mathbf E}}
\def\O{{O}}
\def\N{{\mathcal N}}
\def\o{{\mathcal o}}
\def\s{{\mathbf s}}
\def\CC{{\mathcal C}}
\def\VV{{\mathcal V}}
\def\NN{{\mathcal N}}
\def\KK{{\mathcal K}}
\def\CCC{{C}}
\def\L{{\mathcal L}}
\def\P{{\mathcal P}}
\def\PP{{\mathbf P}}
\def\PPP{{\mathbb P}}
\def\R{{\mathbf R}}
\def\RR{{\mathcal R}}
\def\S{{\mathcal S}}
\def\SS{{\mathbf SS}}
\def\H{{\mathcal H}}
\def\d{{\mathbf d}}
\def\e{{\mathbf e}}
\def\b{{\mathbf b}}
\def\cc{{\mathbf c}}
\def\u{{\mathbf u}}
\def\m{{\mathbf m}}
\def\hatD{{\bf \hat D}}
\def\tildeD{{\mathbf{\tilde{D}}}}
\def\tildeB{{\mathbf{\tilde{B}}}}
\def\tildeC{{\mathbf{\tilde{C}}}}
\def\tildex{{\mathbf{\tilde{x}}}}
\def\tildef{{ \tilde f}}
\def\hatE{{\bf \hat E}}
\def\hate{{\bf \hat e}}
\def\hatd{{\bf \hat d}}
\def\hatf{{ \hat f}}
\def\tilded{{\bf \tilde d}}
\def\Real{{\mathbb R}}
\def\r{{\mathbf r}}
\def\U{{\mathbf U}}
\def\UY{{\mathcal U}}
\def\u{{\mathbf u}}
\def\A{{\mathbf A}}
\def\G{{\mathbb G}}
\def\GG{{\mathcal G}}
\def\AA{{\mathcal A}}
\def\FF{{\mathcal F}}
\def\DD{{\mathcal D}}
\def\XX{{\mathcal X}}
\def\WW{{\mathcal W}}
\def\V{{\mathbf V}}
\def\K{{\mathbf K}}
\def\KK{{\mathcal K}}
\def\T{{\mathbf T}}
\def\TT{{\mathcal T}}
\def\I{{\mathbf I}}
\def\one{{\mathbb 1}}
\def\argmin{\operatornamewithlimits{arg\,min}}
\def\liminf{\operatornamewithlimits{lim\,inf}}
\def\argmax{\operatornamewithlimits{arg\,max}}
\def\Var{\operatorname{Var}}
\def\diag{\operatorname{diag}}
\def\trace{\operatorname{Tr}}
\def\range{\operatorname{range}}
\def\vec{\operatorname{vec}}
\def\sign{\operatorname{sign}}
\def\FL{\operatorname{FL}}
\def\corr{\operatorname{corr}}
\def\cov{\operatorname{cov}}
\def\st{~~\text{s.t.}~~}
\def\defin{\triangleq}
%\def\TODO{ {\color{red} TODO }}
\newcommand{\TODO}[1]{{\textbf{\color{red} TODO: #1}}}

\def\gi{ {\scriptscriptstyle \hspace*{-0.002cm}\mid\hspace*{-0.008cm}}  g}
\def\hi{  {\scriptscriptstyle \hspace*{-0.002cm}\mid\hspace*{-0.008cm}}  h}

%\renewcommand{\cite}{\citep}
\newcommand{\refEq}{\ref}
%\renewenvironment{displaymath}{\begin{equation}}{\end{equation}}
%\renewenvironment{multline*}{\begin{multline}}{\end{multline}}

%\renewcommand{\cite}{\citep}
% \newcommand{\citet}{\emcite}
% \renewenvironment{displaymath}{\begin{equation}}{\end{equation}}
%:w\renewenvironment{multline*}{\begin{multline}}{\end{multline}}

% \newtheorem{example}{Example} 
% 

%\newcommand{\InSet}[1]{\text{\textlbrackdbl} 1;#1 \text{\textrbrackdbl}}
\newcommand{\InSet}[1]{\llbracket 1;#1 \rrbracket}
\newcommand{\IntSet}[1]{\llbracket 1;#1 \rrbracket}
\renewcommand{\bf}{\textbf}

\newcommand{\Norm}[1]{\|#1\|}
\newcommand{\DualNorm}[1]{\|#1\|_{\ast}}
\newcommand{\NormUn}[1]{\|#1\|_1}
\newcommand{\NormDeux}[1]{\|#1\|_2}
\newcommand{\NormInf}[1]{\|#1\|_{\infty}}
\newcommand{\NormFro}[1]{\|#1\|_{\text{F}}}
\newcommand{\NormFrob}[1]{\|#1\|_{\text{F}}}

\newcommand{\INPUT}{ {\bfseries Input: }}

\newcommand{\rred}[1]{{\textcolor{red}{#1}}}
\definecolorset{rgb}{}{}{inriablue,0.35,0.31,0.75}
\newcommand{\bblue}[1]{{\textcolor{inriablue}{#1}}}

\title[Network Flow Algorithms for Structured Sparsity]{\veryHuge{Network Flow Algorithms for Structured Sparsity}}
\author[Julien Mairal]{
   Julien Mairal \and
      Rodolphe Jenatton \and
      Guillaume Obozinski \and
      Francis Bach 
}
\institute[Willow Project Team - INRIA/ENS/CNRS]{~}
% 
\begin{document}
\begin{frame}{} 
\begin{columns}[T]
\hfill
\begin{column}{0.315\paperwidth}
% \begin{block}{One minute overview}
% \begin{itemize}
% \item Solving large-scale \rred{structured sparse} regularized problems.
% \item Use of (accelerated) \rred{proximal gradient methods}.
% \item Proximal operator solved with \rred{network flow optimization}.
% \end{itemize}
% \end{block}
\begin{block}{Problem setting}
\begin{itemize}
\item We are interested in solving
 {\large
 \begin{displaymath}
     \min_{\w \in \Real^p} f(\w) + \lambda \sum_{g \in \GG} \eta_g \|\w_g\|_\infty.
 \end{displaymath}
 }
\item $f$ is a \rred{convex smooth function} (empirical risk, data-fitting term).
\item $p$ can be large (up to $\approx10^6$).
\item $\GG$ is a set of \rred{overlapping} groups of variables.
\end{itemize}
\vspace*{0.5cm}
\textbf{Q: Why such a regularization?}
\begin{itemize}
   \item the $\ell_1$ does not encode structure, just \rred{cardinality}.
   \item variables are sometimes organized into groups that are
   \begin{itemize}
\normalsize
      \item \vspace*{0.2cm}\normalsize{non-overlapping}; (\small{Yuan and Lin '06})
      \item \vspace*{0.2cm}\normalsize{tree-structured}; (\small{Zhao et al '09})
      \item \normalsize{\rred{overlapping}}; \\ (\small{Jenatton et al '09, Jacob et al '09, Huang et al '09, Baraniuk et al '08})
   \end{itemize}
   \item The following penalty {\small(Jenatton et al '09)}
 {\large
\begin{displaymath}
  \Omega(\w) \defin  \sum_{g \in \GG} \eta_g \|\w_g\|_\infty,
\end{displaymath}
}
\hspace*{-0.5cm}encourages variables in a same group to be set to zero~together.
\item \rred{The goal is to encode a-priori knowledge by designing groups}.
\end{itemize}
\end{block}
\begin{block}{Proximal gradient algorithms}
   \begin{itemize}
      \item Generalizes the idea of gradient descent
         \begin{displaymath}
            \begin{split}
               \w^{k+1} & \!\leftarrow \!\argmin_{\w \in \Real^p} {\underbrace{f(\w^k)\! +\! \nabla f(\w^k)^\top(\w-\w^k)}_{\text{linear approximation}}} + {\underbrace{\frac{L}{2}\|\w-\w^k\|_2^2}_{\text{quadratic term}}} \!+\! \lambda\Omega(\w) \\
               & \leftarrow \argmin_{\w \in \Real^p} \frac{1}{2}\|(\w^k - \frac{1}{L}\nabla f(\w^k))-\w\|_2^2 + \frac{\lambda}{L}\Omega(\w). 
            \end{split}
         \end{displaymath}
      \item They require solving efficiently the {\rred{proximal operator $\text{Prox}_{\lambda\Omega}$}}
{\large
         \begin{displaymath}
           \u \to \argmin_{\w \in \Real^p} ~ \frac{1}{2}\|\u-\w\|_2^2 +  \lambda\Omega(\w).
         \end{displaymath}
}
%          \begin{displaymath}
%             \w^\star_i= \sign(\u_i) (\u_i-\lambda)^+.
%          \end{displaymath}
      \item Accelerated versions based on Nesterov first-order method.
      \item For the $\ell_1$-norm, soft-thresholding $$\u_i \to  \sign(\u_i) \max(|\u_i|-\lambda,0).$$
      \item For non-overlapping groups with $\ell_2$-norms, $$\u_g \to \frac{\u_g}{\|\u_g\|_2}\max(\|\u_g\|_2-\lambda,0),$$
      \item For non-overlapping groups with $\ell_\infty$-norms, $$\u_g \to \u_g - \Pi_{\|.\|_1 \leq \lambda}[\u_g].$$
      \item \rred{As soon as the groups overlap, the problem is difficult}.
   \end{itemize}
\end{block}


\end{column}\hfill

\begin{column}{0.315\linewidth}
\begin{block}{Hierarchical norms: Jenatton et al, ICML 2010} 
\begin{itemize}
   \item A set of groups is \rred{tree-structured} if $$\forall g,h \in \GG,~~ g \cap h = \emptyset ~~\text{or}~~ g \subset h ~~\text{or}~~ h \subset g.$$
   \item Order the groups $g_1,\ldots,g_m$ \rred{from the leaves to the root}.
   \item The proximal operator admits a \rred{closed form}: $$\text{Prox}_{\lambda\Omega} = \text{Prox}^{g_1} \circ \ldots \circ \text{Prox}^{g_m}.$$
   \item Sequence of small proximal operators/projections.
   \item \rred{$O(p)$ operations} for $\ell_2$-norms ($O(pd)$ for $\ell_\infty$-norms).
\end{itemize}
\end{block}
\begin{block}{Dual formulation of the proximal operator} 
   {\color{red} Idea: no more overlaps in the dual problem}
\begin{itemize}
\item One dual variable $\xib^g$ per group $g$ in $\GG$.
{\large
\begin{multline*}
\min_{\xib \in \Real^{p \times |\GG|}} \frac{1}{2}\big\| \u - \sum_{g \in \GG} \xib^g \big\|_2^2 \\ \st \forall g \in \GG,~~ \|\xib^g\|_1 \leq \lambda\eta_g ~~\text{and}~~ \xib^g_j = 0 ~~\text{if}~~ j \notin g.
\end{multline*}
}
\item First step: Flip the signs of $\u$, and impose $\xib \geq 0$.
\item This becomes a \rred{quadratic min-cost flow problem}.
\end{itemize}
\end{block}
\begin{block}{Examples of Graph Modelling}
\small{
\begin{figure}[hbtp!]
\tikzstyle{source}=[circle,thick,draw=blue!75,fill=blue!20,minimum size=18mm]
\tikzstyle{sink}=[circle,thick,draw=blue!75,fill=blue!20,minimum size=18mm]
\tikzstyle{group}=[place,thick,draw=red!75,fill=red!20, minimum size=18mm]
\tikzstyle{var}=[rectangle,thick,draw=black!75,fill=black!20,minimum size=14mm]
\def\distnode{3.8cm}
\def\distnodex{5.5cm}
\tikzstyle{every label}=[red]
   \begin{center}
      ~~\subfigure[$\GG\!=\!\{ g\!=\!\{1,2,3\} \}$.]{
      \begin{tikzpicture}[node distance=\distnode,>=stealth',bend angle=45,auto]
         \begin{scope}
            \node [source]   (s)                                    {$s$};
            \node [group]    (g1)  [below of=s]                      {$g$}
            edge  [pre,line width=0.1cm] node[left,xshift=1mm] {$\xib^g_1 \!+\! \xib^g_2 \!+\! \xib^g_3 \!\leq\! \lambda \eta_g$} (s);
            \node [var] (u2) [below of=g1]                    {$\u_2$}
            edge  [pre,line width=0.1cm] node[above, left,xshift=1mm] {$\xib^{g}_2$} (g1);
            \node [var] (u1)  [left of=u2, node distance=\distnodex] {$\u_1$}
            edge  [pre,line width=0.1cm] node[above, left] {$\xib^{g}_1$} (g1);
            \node [var] (u3) [right of=u2, node distance=\distnodex] {$\u_3$}
            edge  [pre,line width=0.1cm] node[above, right] {$\xib^{g}_3$} (g1);
            \node [sink] (si) [below of=u2] {$t$}
            edge [pre,line width=0.1cm] node[above,left] {$\color{red} c_1$} (u1)
            edge [pre,line width=0.1cm] node[above,left,xshift=3mm] {$\color{red} c_2$} (u2)
            edge [pre,line width=0.1cm] node[above,right] {$\color{red} c_3$} (u3);
         \end{scope}
      \end{tikzpicture}\label{subfig:grapha}
      } \hfill 
      \subfigure[$\GG\!=\!\{ g\!=\!\{1,2\},h\!=\!\{2,3\} \}$.]{
      \begin{tikzpicture}[node distance=\distnode,>=stealth',bend angle=45,auto]
         \begin{scope}
            \node [source]   (s)                                    {$s$};
            \node [group]    (g)  [below of=s,xshift=-30mm]                      {$g$}
            edge  [pre,line width=0.1cm] node[left] {$\xib^g_1 \!+ \!\xib^g_2 \!\leq \!\lambda \eta_g$} (s);
            \node [group]    (h)  [below of=s,xshift=30mm]                      {$h$}
            edge  [pre,line width=0.1cm] node[right] {$\xib^h_2 \!+ \!\xib^h_3 \!\leq \!\lambda \eta_h$} (s);
            \node [var] (u2) [below of=g,xshift=30mm]                    {$\u_2$}
            edge  [pre,line width=0.1cm] node[above, right] {$\xib^{h}_2$} (h)
            edge  [pre,line width=0.1cm] node[above, left] {$\xib^{g}_2$} (g);
            \node [var] (u1)  [left of=u2, node distance=\distnodex] {$\u_1$}
            edge  [pre,line width=0.1cm] node[above, left] {$\xib^{g}_1$} (g);
            \node [var] (u3) [right of=u2, node distance=\distnodex] {$\u_3$}
            edge  [pre,line width=0.1cm] node[above, right] {$\xib^{h}_3$} (h);
            \node [sink] (si) [below of=u2] {$t$}
            edge [pre,line width=0.1cm] node[above,left] {$\color{red} c_1$} (u1)
            edge [pre,line width=0.1cm] node[above,left,xshift=1mm] {$\color{red} c_2$} (u2)
            edge [pre,line width=0.1cm] node[above,right] {$\color{red} c_3$} (u3);
         \end{scope}
      \end{tikzpicture}\label{subfig:graphb}
      } ~~~\\
      \subfigure[$\GG\!=\!\{ g\!=\!\{1,2,3\},h\!=\!\{2,3\} \}$.]{
      \begin{tikzpicture}[node distance=\distnode,>=stealth',bend angle=45,auto]
         \begin{scope}
            \node [source]   (s)                                    {$s$};
            \node [group]    (g)  [below of=s]                      {$g$}
            edge  [pre,line width=0.1cm] node[left] {$\xib^g_1 \!+ \!\xib^g_2 \!+ \!\xib^g_3 \!\leq \!\lambda \eta_g$} (s);
            \node [group]    (h)  [right of=g,node distance=\distnodex]                      {$h$}
            edge  [pre,line width=0.1cm] node[right,xshift=-6mm,yshift=6mm] {$\xib^h_2 \!+ \!\xib^h_3 \!\leq \!\lambda \eta_h$} (s);
            \node [var] (u2) [below of=g]                    {$\u_2$}
            edge  [pre,line width=0.1cm] node[above, left] {$\xib^{h}_2$} (h)
            edge  [pre,line width=0.1cm] node[above, left] {$\xib^{g}_2$} (g);
            \node [var] (u1)  [left of=u2, node distance=\distnodex] {$\u_1$}
            edge  [pre,line width=0.1cm] node[above, left] {$\xib^{g}_1$} (g);
            \node [var] (u3) [right of=u2, node distance=\distnodex] {$\u_3$}
            edge  [pre,line width=0.1cm] node[above, right] {$\xib^{g}_3$} (g)
            edge  [pre,line width=0.1cm] node[above, right] {$\xib^{h}_3$} (h);
            \node [sink] (si) [below of=u2] {$t$}
            edge [pre,line width=0.1cm] node[above,left] {$\color{red} c_1$} (u1)
            edge [pre,line width=0.1cm] node[above,left,xshift=1mm] {$\color{red} c_2$} (u2)
            edge [pre,line width=0.1cm] node[above,right] {$\color{red} c_3$} (u3);
         \end{scope}
      \end{tikzpicture}\label{subfig:graphc}
      } \hspace*{-0.5cm}
      \subfigure[$\GG\!=\!\{ g\!=\!\{1\}\cup h,h\!=\!\{2,3\} \}$.]{
      \begin{tikzpicture}[node distance=\distnode,>=stealth',bend angle=45,auto]
         \begin{scope}
            \node [source]   (s)                                    {$s$};
            \node [group]    (g)  [below of=s]                      {$g$}
            edge  [pre,line width=0.1cm] node[left,xshift=4mm] {$\xib^g_1 \!+ \!\xib^g_2 \!+ \!\xib^g_3 \!\leq \!\lambda \eta_g$} (s);
            \node [group]    (h)  [right of=g, node distance=\distnodex]                      {$h$}
            edge  [pre,line width=0.1cm] node[right,yshift=1mm] {$\xib^h_2 \!+ \!\xib^h_3 \!\leq \!\lambda \eta_h$} (s)
            edge  [pre,line width=0.1cm] node[below,yshift=4mm] {$\xib^g_2 \!+ \!\xib^g_3$} (g);
            \node [var] (u2) [below of=g]                    {$\u_2$}
            edge  [pre,line width=0.1cm] node[above,left] {$\xib^g_2\!+\!\xib^{h}_2$} (h);
            \node [var] (u1)  [left of=u2, node distance=\distnodex] {$\u_1$}
            edge  [pre,line width=0.1cm] node[above, left] {$\xib^{g}_1$} (g);
            \node [var] (u3) [right of=u2, node distance=\distnodex] {$\u_3$}
            edge  [pre,line width=0.1cm] node[above, right] {$\xib^g_3\!+\!\xib^{h}_3$} (h);
            \node [sink] (si) [below of=u2] {$t$}
            edge [pre,line width=0.1cm] node[above,left] {$\color{red} c_1$} (u1)
            edge [pre,line width=0.1cm] node[above,left,xshift=1mm] {$\color{red} c_2$} (u2)
            edge [pre,line width=0.1cm] node[above,right] {$\color{red} c_3$} (u3);
         \end{scope}
      \end{tikzpicture} \label{subfig:graphd}
      } 
   \end{center}
   \caption{Graph representation of simple proximal problems.}
\end{figure}
}
\end{block}


\end{column}\hfill
\begin{column}{0.315\linewidth}
\begin{alertblock}{Code available! \textbf{\url{http.//www.di.ens.fr/willow/SPAMS/}}}
\textbf{SPAMS v2.0} (C++ with Matlab interface) implements:
\begin{enumerate}
\item Proximal methods (ISTA/FISTA) with duality gaps for
\begin{itemize}
\item square, logistic and multiclass logistic loss functions;
\item simple norms $\ell_1$, $\ell_2$; mixed-norms $\ell_{1,2}$, $\ell_{1,\infty}$, $\ell_{1,2}+\ell_1$, $\ell_{1,\infty}+\ell_1$;
\item tree-structured $\ell_{2}$- or $\ell_{\infty}$-norms; sum of $\ell_{\infty}$-norms with overlapping groups.
\end{itemize}
\item Various fast sparse solvers:
\begin{itemize}
\item $\ell_1$-solvers: LARS, coordinate descent; Greedy algorithms (OMP),\ldots
\item Various fast Euclidean projections (simplex, $\ell_1$-ball, $\ell_1\!+\!\ell_2^2$,  $\ell_1\!+\!\text{TV}\!+\!\ell_2^2$\ldots).
\end{itemize}

\item Dictionary learning and matrix factorization (NMF, sparse PCA)
\end{enumerate}
\end{alertblock}
\begin{block}{Idea of the algorithm: Divide and conquer}
\begin{enumerate}
   \item Solve a relaxed problem in linear time.
   \item Check whether the solution is feasible with a max-flow.
   \item If yes, it is optimal and stop the algorithm.
   \item If not, find a minimum cut and removes the arcs along the cut.
   \item Recursively process the two subgraphs defined by the cut.
\end{enumerate}
Similar algorithm in submodular functions literature {\small (Groenevelt '91)}.
\end{block}

\begin{block}{Experiments}
{\bfseries Background subtraction}~{\small (Cehver '08, Huang '09, Wright '08)}
\begin{itemize}
\item $\y = \X\w + \e$; $\X$, background images; $\e$ is sparse and structured.
\item $p \approx 60\,000$; $|\GG| \approx 120\,000$; $|E| \approx 660\,000$; $1.5sec / \text{prox}$
\end{itemize}
   \begin{figure}
      \centering
%      \includegraphics[width=6.6cm]{images/original_trees.png} \hfill
%      \includegraphics[width=6.6cm]{images/background_trees_struct.png}\hfill
%      \includegraphics[width=6.6cm]{images/foreground_trees_L1.png}\hfill 
%      \includegraphics[width=6.6cm]{images/foreground_trees_struct.png}\hfill 
%      \includegraphics[width=6.6cm]{images/foreground_trees_struct_BIS.png} \\
%     \vspace*{0.2cm}
%       \includegraphics[width=6.6cm]{images/original_video_boot.png}\hfill
%       \includegraphics[width=6.6cm]{images/background_video_boot_struct.png}\hfill
%       \includegraphics[width=6.6cm]{images/foreground_video_boot_L1.png}\hfill
%       \includegraphics[width=6.6cm]{images/foreground_video_boot_struct.png}\hfill 
%       \includegraphics[width=6.6cm]{images/foreground_video_boot_struct_BIS.png} \\
    \caption{{\small Original, est. background, est. foreground with $\ell_1$, with $\ell_1+\Omega$.}}
   \end{figure}
\vspace*{-0.3cm}
{\bfseries Multi-task learning of hierarchical structures}
   \begin{displaymath}
    \min_{\X,\W}
    \frac{1}{n}\sum_{i=1}^n\!\Big[\frac{1}{2} \|\y^i-\X\w^i\|_2^2 + \lambda_1 \Omega_{\text{tree}}(\w^i)\Big]\!+\!\lambda_2\Omega_{\text{joint}}(\W),\ \text{s.t.}\
    \forall j,~~\|\x^j\|_2\leq 1, 
\end{displaymath}
$n=10\,000$ image patches; $p,|\GG| \approx 4\,000\,000$; $|E|\approx 12\,000\,000$
\begin{figure}[hbtp]
   \centering
%   \includegraphics[width=0.42\linewidth]{images/tree2.png}\hfill
%   \includegraphics[width=0.54\linewidth]{images/tree_denois.png} 
   \caption{Example of hierarchy - Mean square error versus dictionary size. 
   } \label{fig:tree}
\end{figure}
\end{block}
\end{column}
\end{columns}

\end{frame}
\end{document}

