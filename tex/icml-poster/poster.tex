\documentclass[final]{beamer} 
\usetheme[headheight=10cm,footheight=5cm]{boxes}
\usetheme{toby}
%\usepackage{times}
\usepackage{etex}
\usepackage{amsmath,amssymb}
\usepackage[latin1]{inputenc}
\usepackage[english]{babel}
\usepackage{helvet}
\usepackage[T1]{fontenc}
\usepackage{textcomp}
\usepackage[orientation=landscape,size=a0,scale=1.4,debug]{beamerposter}   % e.g. for DIN-A0 poster
\usepackage{exscale}
\usepackage{pst-plot}
\usepackage{pstricks-add}
\usepackage{epsfig}
\usepackage{eulervm}
\usepackage{tikz}
\usepackage{array}
\usepackage{subfigure}
\usepackage{graphicx}
  
% \usepackage[size=custom,width=3.0,height=120,scale=2,debug]{beamerposter}  % e.g. custom size poster

\usepackage{beamerthemetoby}

\def\x{{\mathbf x}}
\def\z{{\mathbf z}}
\def\l{{\ell}}
\def\1{{\mathbf 1}}
\def\0{{\mathbf 0}}
\def\a{{\mathbf a}}
%\def\c{{\mathbf c}}
\def\p{{\mathbf p}}
\def\v{{\mathbf v}}
\def\n{{\mathbf n}}
\def\X{{\mathbf X}}
\def\Z{{\mathbf Z}}
\def\eg{e.g.}
\def\ZZ{{\mathcal Z}}
\def\hatx{{\mathbf {\hat x}}}
\def\hatf{{ \hat f}}
\def\hatw{{\bf \hat w}}
\def\hati{{\hat{\imath}}}
\def\tildej{\tilde{\jmath}}
\def\tildel{\tilde{\ell}}
\def\hatX{{\bf \hat X}}
\def\hatb{{\hat b}}
\def\ae{\text{~~a.e.}}
\def\as{\text{~~a.s.}}
\def\hatalpha{\mbox{\boldmath${\hat \alpha}$}}
\def\tildealpha{\mbox{\boldmath${\tilde \alpha}$}}
\def\tildebeta{\mbox{\boldmath${\tilde \beta}$}}
\def\hatDelta{\mbox{\boldmath${\hat \Delta}$}}
\def\Lambdab{{\boldsymbol\Lambda}}
\def\lambdab{{\boldsymbol\lambda}}
\def\kappab{{\boldsymbol\kappa}}
\def\rhob{{\boldsymbol\rho}}
\def\taub{{\boldsymbol\tau}}

\def\xib{{\boldsymbol\xi}}
\def\xibbar{\tilde{\boldsymbol{\xi}}}
\def\nub{{\boldsymbol\nu}}
\def\Gammab{{\boldsymbol\Gamma}}
\def\Thetab{{\boldsymbol\Theta}}
\def\Sigmab{{\boldsymbol\Sigma}}
\def\Deltab{{\boldsymbol\Delta}}
\def\betab{{\boldsymbol\beta}}
\def\epsilonb{{\boldsymbol\varepsilon}}
\def\varepsilonb{{\boldsymbol\varepsilon}}
\def\alphab{{\boldsymbol\alpha}}
\def\thetab{{\boldsymbol\theta}}
\def\gammab{{\boldsymbol\gamma}}
\def\alphabtilde{{\boldsymbol\tilde{\alpha}}}
\def\hatlambdab{{\boldsymbol{\hat\lambda}}}
\def\Gammab{{\boldsymbol\Gamma}}
\def\tildeGammab{{\bf \tilde{\boldsymbol\Gamma}}}
\def\y{{\mathbf y}}
\def\FFF{\text{F}}
\def\w{{\mathbf w}}
\def\vw{{\mathbf w}}
\def\D{{\mathbf D}}
\def\mX{{\mathbf D}}
\def\Y{{\mathbf Y}}
\def\YY{{\mathcal Y}}
\def\M{{\mathbf M}}
\def\B{{\mathbf B}}
\def\Q{{\mathbf Q}}
\def\C{{\mathbf C}}
\def\W{{\mathbf W}}
\def\d{{\mathbf d}}
\def\E{{\mathbb E}}
\def\PPP{{\mathbb E}}
\def\EE{{\mathbf E}}
\def\O{{O}}
\def\N{{\mathcal N}}
\def\o{{\mathcal o}}
\def\s{{\mathbf s}}
\def\CC{{\mathcal C}}
\def\VV{{\mathcal V}}
\def\NN{{\mathcal N}}
\def\KK{{\mathcal K}}
\def\CCC{{C}}
\def\L{{\mathcal L}}
\def\P{{\mathcal P}}
\def\PP{{\mathbf P}}
\def\PPP{{\mathbb P}}
\def\R{{\mathbf R}}
\def\RR{{\mathcal R}}
\def\S{{\mathcal S}}
\def\SS{{\mathbf SS}}
\def\H{{\mathcal H}}
\def\d{{\mathbf d}}
\def\e{{\mathbf e}}
\def\b{{\mathbf b}}
\def\cc{{\mathbf c}}
\def\u{{\mathbf u}}
\def\m{{\mathbf m}}
\def\hatD{{\bf \hat D}}
\def\tildeD{{\mathbf{\tilde{D}}}}
\def\tildeB{{\mathbf{\tilde{B}}}}
\def\tildeC{{\mathbf{\tilde{C}}}}
\def\tildex{{\mathbf{\tilde{x}}}}
\def\tildef{{ \tilde f}}
\def\hatE{{\bf \hat E}}
\def\hate{{\bf \hat e}}
\def\hatd{{\bf \hat d}}
\def\hatf{{ \hat f}}
\def\tilded{{\bf \tilde d}}
\def\Real{{\mathbb R}}
\def\r{{\mathbf r}}
\def\U{{\mathbf U}}
\def\UY{{\mathcal U}}
\def\u{{\mathbf u}}
\def\A{{\mathbf A}}
\def\G{{\mathbb G}}
\def\GG{{\mathcal G}}
\def\AA{{\mathcal A}}
\def\FF{{\mathcal F}}
\def\DD{{\mathcal D}}
\def\XX{{\mathcal X}}
\def\WW{{\mathcal W}}
\def\V{{\mathbf V}}
\def\K{{\mathbf K}}
\def\KK{{\mathcal K}}
\def\T{{\mathbf T}}
\def\TT{{\mathcal T}}
\def\I{{\mathbf I}}
\def\one{{\mathbb 1}}
\def\argmin{\operatornamewithlimits{arg\,min}}
\def\liminf{\operatornamewithlimits{lim\,inf}}
\def\argmax{\operatornamewithlimits{arg\,max}}
\def\Var{\operatorname{Var}}
\def\diag{\operatorname{diag}}
\def\trace{\operatorname{Tr}}
\def\range{\operatorname{range}}
\def\vec{\operatorname{vec}}
\def\sign{\operatorname{sign}}
\def\FL{\operatorname{FL}}
\def\corr{\operatorname{corr}}
\def\cov{\operatorname{cov}}
\def\st{~~\text{s.t.}~~}
\def\defin{\triangleq}
%\def\TODO{ {\color{red} TODO }}
\newcommand{\TODO}[1]{{\textbf{\color{red} TODO: #1}}}

\def\gi{ {\scriptscriptstyle \hspace*{-0.002cm}\mid\hspace*{-0.008cm}}  g}
\def\hi{  {\scriptscriptstyle \hspace*{-0.002cm}\mid\hspace*{-0.008cm}}  h}

%\renewcommand{\cite}{\citep}
\newcommand{\refEq}{\ref}
%\renewenvironment{displaymath}{\begin{equation}}{\end{equation}}
%\renewenvironment{multline*}{\begin{multline}}{\end{multline}}

%\renewcommand{\cite}{\citep}
% \newcommand{\citet}{\emcite}
% \renewenvironment{displaymath}{\begin{equation}}{\end{equation}}
%:w\renewenvironment{multline*}{\begin{multline}}{\end{multline}}

% \newtheorem{example}{Example} 
% 

%\newcommand{\InSet}[1]{\text{\textlbrackdbl} 1;#1 \text{\textrbrackdbl}}
\newcommand{\InSet}[1]{\llbracket 1;#1 \rrbracket}
\newcommand{\IntSet}[1]{\llbracket 1;#1 \rrbracket}
\renewcommand{\bf}{\textbf}

\newcommand{\Norm}[1]{\|#1\|}
\newcommand{\DualNorm}[1]{\|#1\|_{\ast}}
\newcommand{\NormUn}[1]{\|#1\|_1}
\newcommand{\NormDeux}[1]{\|#1\|_2}
\newcommand{\NormInf}[1]{\|#1\|_{\infty}}
\newcommand{\NormFro}[1]{\|#1\|_{\text{F}}}
\newcommand{\NormFrob}[1]{\|#1\|_{\text{F}}}

\newcommand{\INPUT}{ {\bfseries Input: }}

\newcommand{\rred}[1]{{\textcolor{red}{#1}}}
\definecolorset{rgb}{}{}{inriablue,0.35,0.31,0.75}
\newcommand{\bblue}[1]{{\textcolor{inriablue}{#1}}}

\title[]{\veryHuge{Clusterpath: an Algorithm for Clustering using Convex Fusion Penalties}}
\author[]{Toby Dylan Hocking \and
      Armand Joulin \and
      Francis Bach \and
      Jean-Philippe Vert}
    \institute[Sierra]{~}%DONT DELETE THIS
\begin{document}
\begin{frame}{} 
\begin{columns}[T]
\hfill
\begin{column}{0.315\paperwidth}
% \begin{block}{One minute overview}
% \begin{itemize}
% \item Solving large-scale \rred{structured sparse} regularized problems.
% \item Use of (accelerated) \rred{proximal gradient methods}.
% \item Proximal operator solved with \rred{network flow optimization}.
% \end{itemize}
% \end{block}
\begin{block}{Problem setting}
\begin{itemize}
\item We are interested in solving
 {\large
 \begin{displaymath}
     \min_{\w \in \Real^p} f(\w) + \lambda \sum_{g \in \GG} \eta_g \|\w_g\|_\infty.
 \end{displaymath}
 }
\item $f$ is a \rred{convex smooth function} (empirical risk, data-fitting term).
\item $p$ can be large (up to $\approx10^6$).
\item $\GG$ is a set of \rred{overlapping} groups of variables.
\end{itemize}
\vspace*{0.5cm}
\textbf{Q: Why such a regularization?}
\begin{itemize}
   \item the $\ell_1$ does not encode structure, just \rred{cardinality}.
   \item variables are sometimes organized into groups that are
   \begin{itemize}
\normalsize
      \item \vspace*{0.2cm}\normalsize{non-overlapping}; (\small{Yuan and Lin '06})
      \item \vspace*{0.2cm}\normalsize{tree-structured}; (\small{Zhao et al '09})
      \item \normalsize{\rred{overlapping}}; \\ (\small{Jenatton et al '09, Jacob et al '09, Huang et al '09, Baraniuk et al '08})
   \end{itemize}
   \item The following penalty {\small(Jenatton et al '09)}
 {\large
\begin{displaymath}
  \Omega(\w) \defin  \sum_{g \in \GG} \eta_g \|\w_g\|_\infty,
\end{displaymath}
}
\hspace*{-0.5cm}encourages variables in a same group to be set to zero~together.
\item \rred{The goal is to encode a-priori knowledge by designing groups}.
\end{itemize}
\end{block}
\begin{block}{Proximal gradient algorithms}
   \begin{itemize}
      \item Generalizes the idea of gradient descent
         \begin{displaymath}
            \begin{split}
               \w^{k+1} & \!\leftarrow \!\argmin_{\w \in \Real^p} {\underbrace{f(\w^k)\! +\! \nabla f(\w^k)^\top(\w-\w^k)}_{\text{linear approximation}}} + {\underbrace{\frac{L}{2}\|\w-\w^k\|_2^2}_{\text{quadratic term}}} \!+\! \lambda\Omega(\w) \\
               & \leftarrow \argmin_{\w \in \Real^p} \frac{1}{2}\|(\w^k - \frac{1}{L}\nabla f(\w^k))-\w\|_2^2 + \frac{\lambda}{L}\Omega(\w). 
            \end{split}
         \end{displaymath}
      \item They require solving efficiently the {\rred{proximal operator $\text{Prox}_{\lambda\Omega}$}}
{\large
         \begin{displaymath}
           \u \to \argmin_{\w \in \Real^p} ~ \frac{1}{2}\|\u-\w\|_2^2 +  \lambda\Omega(\w).
         \end{displaymath}
}
%          \begin{displaymath}
%             \w^\star_i= \sign(\u_i) (\u_i-\lambda)^+.
%          \end{displaymath}
      \item Accelerated versions based on Nesterov first-order method.
      \item For the $\ell_1$-norm, soft-thresholding $$\u_i \to  \sign(\u_i) \max(|\u_i|-\lambda,0).$$
      \item For non-overlapping groups with $\ell_2$-norms, $$\u_g \to \frac{\u_g}{\|\u_g\|_2}\max(\|\u_g\|_2-\lambda,0),$$
      \item For non-overlapping groups with $\ell_\infty$-norms, $$\u_g \to \u_g - \Pi_{\|.\|_1 \leq \lambda}[\u_g].$$
      \item \rred{As soon as the groups overlap, the problem is difficult}.
   \end{itemize}
\end{block}


\end{column}\hfill

\begin{column}{0.315\linewidth}
\begin{block}{Hierarchical norms: Jenatton et al, ICML 2010} 
\begin{itemize}
   \item A set of groups is \rred{tree-structured} if $$\forall g,h \in \GG,~~ g \cap h = \emptyset ~~\text{or}~~ g \subset h ~~\text{or}~~ h \subset g.$$
   \item Order the groups $g_1,\ldots,g_m$ \rred{from the leaves to the root}.
   \item The proximal operator admits a \rred{closed form}: $$\text{Prox}_{\lambda\Omega} = \text{Prox}^{g_1} \circ \ldots \circ \text{Prox}^{g_m}.$$
   \item Sequence of small proximal operators/projections.
   \item \rred{$O(p)$ operations} for $\ell_2$-norms ($O(pd)$ for $\ell_\infty$-norms).
\end{itemize}
\end{block}
\begin{block}{Dual formulation of the proximal operator} 
   {\color{red} Idea: no more overlaps in the dual problem}
\begin{itemize}
\item One dual variable $\xib^g$ per group $g$ in $\GG$.
{\large
\begin{multline*}
\min_{\xib \in \Real^{p \times |\GG|}} \frac{1}{2}\big\| \u - \sum_{g \in \GG} \xib^g \big\|_2^2 \\ \st \forall g \in \GG,~~ \|\xib^g\|_1 \leq \lambda\eta_g ~~\text{and}~~ \xib^g_j = 0 ~~\text{if}~~ j \notin g.
\end{multline*}
}
\item First step: Flip the signs of $\u$, and impose $\xib \geq 0$.
\item This becomes a \rred{quadratic min-cost flow problem}.
\end{itemize}
\end{block}

\begin{block}{Examples of Graph Modelling}
\small{}
\end{block}


\end{column}\hfill
\begin{column}{0.315\linewidth}
\begin{alertblock}{Code!
    \textbf{\url{http.//clusterpath.r-forge.r-project.org/}}}
\begin{itemize}
\item Dedicated C++ optimization algorithms with R interface.
  \begin{itemize}
  \item Path algorithm for $\ell_1$ norm with identity weights.
  \item Active-set descent algorithm for $\ell_2$ problem.
  \end{itemize}
\item R interface to Python \texttt{cvxmod} clusterpath solver.
\item Clusterpath visualizations in 2d, 3d, and animations.
\item Coming soon: picking the number of clusters automatically!
\end{itemize}
\end{alertblock}


\begin{block}{Outline of the $\ell_1$ path algorithm}
\begin{enumerate}
\item For $\lambda=0$ the solution $\alpha=X$ is optimal. We
  initialize the clusters $C_i = \{i\}$ and coefficients $\alpha_i =
  X_i$ for all $i$.
\item As $\lambda$ increases, the solutions will follow straight
  lines until they hit.
\item Taking the derivative of the optimality condition with respect
  to $\lambda$ leads to the following expression for the velocity of
  the initial clusters:
$$v_i = \sum_{j\neq i}w_{ij}\sign(\alpha_j-\alpha_i)$$
\item When 2 clusters $C_1$ and $C_2$ hit, they will merge to form a
  new cluster $C = C_1\cup C_2$ and take a new velocity:
$$v_C = \frac{
|C_1|v_1 + |C_2|v_2
}{
|C_1|+|C_2|
}$$
\item Stop when all the points merge at the mean $\overline X$.
\item Combine dimensions using $\lambda$ values.
\end{enumerate}

\end{block}

% Created by tikzDevice version 0.6.1 on 2011-06-26 17:49:00
% !TEX encoding = UTF-8 Unicode
\begin{tikzpicture}[x=1pt,y=1pt]
\definecolor[named]{drawColor}{rgb}{0.00,0.00,0.00}
\definecolor[named]{fillColor}{rgb}{1.00,1.00,1.00}
\fill[color=fillColor,] (0,0) rectangle (722.70,433.62);
\begin{scope}
\path[clip] (  0.00,  0.00) rectangle (722.70,433.62);
\end{scope}
\begin{scope}
\path[clip] (  0.00,  0.00) rectangle (722.70,433.62);
\end{scope}
\begin{scope}
\path[clip] (  0.00,  0.00) rectangle (722.70,433.62);
\end{scope}
\begin{scope}
\path[clip] (  0.00,  0.00) rectangle (722.70,433.62);
\end{scope}
\begin{scope}
\path[clip] (  0.00,  0.00) rectangle (722.70,433.62);
\end{scope}
\begin{scope}
\path[clip] (  0.00,  0.00) rectangle (722.70,433.62);
\end{scope}
\begin{scope}
\path[clip] (  0.00,  0.00) rectangle (722.70,433.62);
\end{scope}
\begin{scope}
\path[clip] (  0.00,  0.00) rectangle (722.70,433.62);
\end{scope}
\begin{scope}
\path[clip] (  0.00,  0.00) rectangle (722.70,433.62);
\definecolor[named]{fillColor}{rgb}{1.00,1.00,1.00}

\draw[fill=fillColor,draw opacity=0.00,] (  0.00,  0.00) rectangle (722.70,433.62);
\end{scope}
\begin{scope}
\path[clip] (  0.00,  0.00) rectangle (722.70,433.62);
\definecolor[named]{drawColor}{rgb}{0.00,0.00,0.00}

\node[color=drawColor,anchor=base,inner sep=0pt, outer sep=0pt, scale=  0.60] at (356.02,392.92) {$\ell_1$ clusterpath of 10 points in 2d%
};
\end{scope}
\begin{scope}
\path[clip] (  0.00,  0.00) rectangle (722.70,433.62);
\definecolor[named]{drawColor}{rgb}{0.00,0.00,0.00}

\node[color=drawColor,anchor=base,inner sep=0pt, outer sep=0pt, scale=  0.42] at (356.02,  7.23) {$\alpha^2$\hspace{2in}%
};
\end{scope}
\begin{scope}
\path[clip] (  0.00,  0.00) rectangle (722.70,433.62);
\definecolor[named]{drawColor}{rgb}{0.00,0.00,0.00}

\node[rotate= 90.00,color=drawColor,anchor=base,inner sep=0pt, outer sep=0pt, scale=  0.50] at ( 25.48,208.89) {$\alpha^1$%
};
\end{scope}
\begin{scope}
\path[clip] (  0.00,  0.00) rectangle (722.70,433.62);
\end{scope}
\begin{scope}
\path[clip] ( 44.91,385.70) rectangle ( 72.08,385.70);
\end{scope}
\begin{scope}
\path[clip] (  0.00,  0.00) rectangle (722.70,433.62);
\end{scope}
\begin{scope}
\path[clip] ( 44.91,385.70) rectangle ( 72.08,385.70);
\end{scope}
\begin{scope}
\path[clip] (  0.00,  0.00) rectangle (722.70,433.62);
\end{scope}
\begin{scope}
\path[clip] (  0.00,  0.00) rectangle (722.70,433.62);
\end{scope}
\begin{scope}
\path[clip] (  0.00,  0.00) rectangle (722.70,433.62);
\end{scope}
\begin{scope}
\path[clip] ( 44.91, 55.41) rectangle ( 72.08, 55.41);
\end{scope}
\begin{scope}
\path[clip] (  0.00,  0.00) rectangle (722.70,433.62);
\end{scope}
\begin{scope}
\path[clip] ( 44.91, 32.08) rectangle ( 72.08, 55.41);
\end{scope}
\begin{scope}
\path[clip] (  0.00,  0.00) rectangle (722.70,433.62);
\end{scope}
\begin{scope}
\path[clip] ( 44.91, 32.08) rectangle ( 72.08, 32.08);
\end{scope}
\begin{scope}
\path[clip] (  0.00,  0.00) rectangle (722.70,433.62);
\end{scope}
\begin{scope}
\path[clip] ( 72.08,385.70) rectangle ( 72.08,385.70);
\end{scope}
\begin{scope}
\path[clip] (  0.00,  0.00) rectangle (722.70,433.62);
\end{scope}
\begin{scope}
\path[clip] ( 72.08,385.70) rectangle ( 72.08,385.70);
\end{scope}
\begin{scope}
\path[clip] (  0.00,  0.00) rectangle (722.70,433.62);
\end{scope}
\begin{scope}
\path[clip] ( 72.08, 55.41) rectangle ( 72.08,385.70);
\end{scope}
\begin{scope}
\path[clip] (  0.00,  0.00) rectangle (722.70,433.62);
\end{scope}
\begin{scope}
\path[clip] ( 72.08, 55.41) rectangle ( 72.08, 55.41);
\end{scope}
\begin{scope}
\path[clip] (  0.00,  0.00) rectangle (722.70,433.62);
\end{scope}
\begin{scope}
\path[clip] ( 72.08, 32.08) rectangle ( 72.08, 55.41);
\end{scope}
\begin{scope}
\path[clip] (  0.00,  0.00) rectangle (722.70,433.62);
\end{scope}
\begin{scope}
\path[clip] ( 72.08, 32.08) rectangle ( 72.08, 32.08);
\end{scope}
\begin{scope}
\path[clip] (  0.00,  0.00) rectangle (722.70,433.62);
\end{scope}
\begin{scope}
\path[clip] ( 72.08,385.70) rectangle (667.14,385.70);
\end{scope}
\begin{scope}
\path[clip] (  0.00,  0.00) rectangle (722.70,433.62);
\end{scope}
\begin{scope}
\path[clip] ( 72.08,385.70) rectangle (667.14,385.70);
\end{scope}
\begin{scope}
\path[clip] (  0.00,  0.00) rectangle (722.70,433.62);
\end{scope}
\begin{scope}
\path[clip] ( 72.08, 55.41) rectangle (667.14,385.70);
\end{scope}
\begin{scope}
\path[clip] (  0.00,  0.00) rectangle (722.70,433.62);
\end{scope}
\begin{scope}
\path[clip] ( 72.08, 55.41) rectangle (667.14, 55.41);
\end{scope}
\begin{scope}
\path[clip] (  0.00,  0.00) rectangle (722.70,433.62);
\end{scope}
\begin{scope}
\path[clip] (  0.00,  0.00) rectangle (722.70,433.62);
\end{scope}
\begin{scope}
\path[clip] (  0.00,  0.00) rectangle (722.70,433.62);
\end{scope}
\begin{scope}
\path[clip] ( 72.08, 32.08) rectangle (667.14, 32.08);
\end{scope}
\begin{scope}
\path[clip] (  0.00,  0.00) rectangle (722.70,433.62);
\end{scope}
\begin{scope}
\path[clip] (667.14,385.70) rectangle (667.14,385.70);
\end{scope}
\begin{scope}
\path[clip] (  0.00,  0.00) rectangle (722.70,433.62);
\end{scope}
\begin{scope}
\path[clip] (667.14,385.70) rectangle (667.14,385.70);
\end{scope}
\begin{scope}
\path[clip] (  0.00,  0.00) rectangle (722.70,433.62);
\end{scope}
\begin{scope}
\path[clip] (667.14, 55.41) rectangle (667.14,385.70);
\end{scope}
\begin{scope}
\path[clip] (  0.00,  0.00) rectangle (722.70,433.62);
\end{scope}
\begin{scope}
\path[clip] (667.14, 55.41) rectangle (667.14, 55.41);
\end{scope}
\begin{scope}
\path[clip] (  0.00,  0.00) rectangle (722.70,433.62);
\end{scope}
\begin{scope}
\path[clip] (667.14, 32.08) rectangle (667.14, 55.41);
\end{scope}
\begin{scope}
\path[clip] (  0.00,  0.00) rectangle (722.70,433.62);
\end{scope}
\begin{scope}
\path[clip] (667.14, 32.08) rectangle (667.14, 32.08);
\end{scope}
\begin{scope}
\path[clip] (  0.00,  0.00) rectangle (722.70,433.62);
\end{scope}
\begin{scope}
\path[clip] (667.14,385.70) rectangle (667.14,385.70);
\end{scope}
\begin{scope}
\path[clip] (  0.00,  0.00) rectangle (722.70,433.62);
\end{scope}
\begin{scope}
\path[clip] (667.14,385.70) rectangle (667.14,385.70);
\end{scope}
\begin{scope}
\path[clip] (  0.00,  0.00) rectangle (722.70,433.62);
\end{scope}
\begin{scope}
\path[clip] (667.14, 55.41) rectangle (667.14,385.70);
\end{scope}
\begin{scope}
\path[clip] (  0.00,  0.00) rectangle (722.70,433.62);
\end{scope}
\begin{scope}
\path[clip] (667.14, 55.41) rectangle (667.14, 55.41);
\end{scope}
\begin{scope}
\path[clip] (  0.00,  0.00) rectangle (722.70,433.62);
\end{scope}
\begin{scope}
\path[clip] (667.14, 32.08) rectangle (667.14, 55.41);
\end{scope}
\begin{scope}
\path[clip] (  0.00,  0.00) rectangle (722.70,433.62);
\end{scope}
\begin{scope}
\path[clip] (667.14, 32.08) rectangle (667.14, 32.08);
\end{scope}
\begin{scope}
\path[clip] (  0.00,  0.00) rectangle (722.70,433.62);
\end{scope}
\begin{scope}
\path[clip] (667.14,385.70) rectangle (667.14,385.70);
\end{scope}
\begin{scope}
\path[clip] (  0.00,  0.00) rectangle (722.70,433.62);
\end{scope}
\begin{scope}
\path[clip] (667.14,385.70) rectangle (667.14,385.70);
\end{scope}
\begin{scope}
\path[clip] (  0.00,  0.00) rectangle (722.70,433.62);
\end{scope}
\begin{scope}
\path[clip] (667.14, 55.41) rectangle (667.14,385.70);
\end{scope}
\begin{scope}
\path[clip] (  0.00,  0.00) rectangle (722.70,433.62);
\end{scope}
\begin{scope}
\path[clip] (667.14, 55.41) rectangle (667.14, 55.41);
\end{scope}
\begin{scope}
\path[clip] (  0.00,  0.00) rectangle (722.70,433.62);
\end{scope}
\begin{scope}
\path[clip] (667.14, 32.08) rectangle (667.14, 55.41);
\end{scope}
\begin{scope}
\path[clip] (  0.00,  0.00) rectangle (722.70,433.62);
\end{scope}
\begin{scope}
\path[clip] (667.14, 32.08) rectangle (667.14, 32.08);
\end{scope}
\begin{scope}
\path[clip] (  0.00,  0.00) rectangle (722.70,433.62);
\end{scope}
\begin{scope}
\path[clip] ( 44.91,385.70) rectangle ( 72.08,385.70);
\end{scope}
\begin{scope}
\path[clip] (  0.00,  0.00) rectangle (722.70,433.62);
\end{scope}
\begin{scope}
\path[clip] ( 44.91,385.70) rectangle ( 72.08,385.70);
\end{scope}
\begin{scope}
\path[clip] (  0.00,  0.00) rectangle (722.70,433.62);
\end{scope}
\begin{scope}
\path[clip] (  0.00,  0.00) rectangle (722.70,433.62);
\definecolor[named]{drawColor}{rgb}{0.50,0.50,0.50}

\node[color=drawColor,anchor=base east,inner sep=0pt, outer sep=0pt, scale=  0.40] at ( 51.92, 83.34) {-0.8%
};

\node[color=drawColor,anchor=base east,inner sep=0pt, outer sep=0pt, scale=  0.40] at ( 51.92,123.12) {-0.6%
};

\node[color=drawColor,anchor=base east,inner sep=0pt, outer sep=0pt, scale=  0.40] at ( 51.92,162.89) {-0.4%
};

\node[color=drawColor,anchor=base east,inner sep=0pt, outer sep=0pt, scale=  0.40] at ( 51.92,202.67) {-0.2%
};

\node[color=drawColor,anchor=base east,inner sep=0pt, outer sep=0pt, scale=  0.40] at ( 51.92,242.44) {0.0%
};

\node[color=drawColor,anchor=base east,inner sep=0pt, outer sep=0pt, scale=  0.40] at ( 51.92,282.22) {0.2%
};

\node[color=drawColor,anchor=base east,inner sep=0pt, outer sep=0pt, scale=  0.40] at ( 51.92,321.99) {0.4%
};

\node[color=drawColor,anchor=base east,inner sep=0pt, outer sep=0pt, scale=  0.40] at ( 51.92,361.77) {0.6%
};
\end{scope}
\begin{scope}
\path[clip] (  0.00,  0.00) rectangle (722.70,433.62);
\definecolor[named]{drawColor}{rgb}{0.50,0.50,0.50}

\draw[color=drawColor,line width= 0.6pt,line cap=round,line join=round,fill opacity=0.00,] ( 67.82, 84.86) -- ( 72.08, 84.86);

\draw[color=drawColor,line width= 0.6pt,line cap=round,line join=round,fill opacity=0.00,] ( 67.82,124.64) -- ( 72.08,124.64);

\draw[color=drawColor,line width= 0.6pt,line cap=round,line join=round,fill opacity=0.00,] ( 67.82,164.41) -- ( 72.08,164.41);

\draw[color=drawColor,line width= 0.6pt,line cap=round,line join=round,fill opacity=0.00,] ( 67.82,204.19) -- ( 72.08,204.19);

\draw[color=drawColor,line width= 0.6pt,line cap=round,line join=round,fill opacity=0.00,] ( 67.82,243.96) -- ( 72.08,243.96);

\draw[color=drawColor,line width= 0.6pt,line cap=round,line join=round,fill opacity=0.00,] ( 67.82,283.74) -- ( 72.08,283.74);

\draw[color=drawColor,line width= 0.6pt,line cap=round,line join=round,fill opacity=0.00,] ( 67.82,323.51) -- ( 72.08,323.51);

\draw[color=drawColor,line width= 0.6pt,line cap=round,line join=round,fill opacity=0.00,] ( 67.82,363.29) -- ( 72.08,363.29);
\end{scope}
\begin{scope}
\path[clip] (  0.00,  0.00) rectangle (722.70,433.62);
\end{scope}
\begin{scope}
\path[clip] (  0.00,  0.00) rectangle (722.70,433.62);
\end{scope}
\begin{scope}
\path[clip] (  0.00,  0.00) rectangle (722.70,433.62);
\end{scope}
\begin{scope}
\path[clip] ( 44.91, 55.41) rectangle ( 72.08, 55.41);
\end{scope}
\begin{scope}
\path[clip] (  0.00,  0.00) rectangle (722.70,433.62);
\end{scope}
\begin{scope}
\path[clip] ( 44.91, 32.08) rectangle ( 72.08, 55.41);
\end{scope}
\begin{scope}
\path[clip] (  0.00,  0.00) rectangle (722.70,433.62);
\end{scope}
\begin{scope}
\path[clip] ( 44.91, 32.08) rectangle ( 72.08, 32.08);
\end{scope}
\begin{scope}
\path[clip] (  0.00,  0.00) rectangle (722.70,433.62);
\end{scope}
\begin{scope}
\path[clip] ( 72.08,385.70) rectangle ( 72.08,385.70);
\end{scope}
\begin{scope}
\path[clip] (  0.00,  0.00) rectangle (722.70,433.62);
\end{scope}
\begin{scope}
\path[clip] ( 72.08,385.70) rectangle ( 72.08,385.70);
\end{scope}
\begin{scope}
\path[clip] (  0.00,  0.00) rectangle (722.70,433.62);
\end{scope}
\begin{scope}
\path[clip] ( 72.08, 55.41) rectangle ( 72.08,385.70);
\end{scope}
\begin{scope}
\path[clip] (  0.00,  0.00) rectangle (722.70,433.62);
\end{scope}
\begin{scope}
\path[clip] ( 72.08, 55.41) rectangle ( 72.08, 55.41);
\end{scope}
\begin{scope}
\path[clip] (  0.00,  0.00) rectangle (722.70,433.62);
\end{scope}
\begin{scope}
\path[clip] ( 72.08, 32.08) rectangle ( 72.08, 55.41);
\end{scope}
\begin{scope}
\path[clip] (  0.00,  0.00) rectangle (722.70,433.62);
\end{scope}
\begin{scope}
\path[clip] ( 72.08, 32.08) rectangle ( 72.08, 32.08);
\end{scope}
\begin{scope}
\path[clip] (  0.00,  0.00) rectangle (722.70,433.62);
\end{scope}
\begin{scope}
\path[clip] ( 72.08,385.70) rectangle (667.14,385.70);
\end{scope}
\begin{scope}
\path[clip] (  0.00,  0.00) rectangle (722.70,433.62);
\end{scope}
\begin{scope}
\path[clip] ( 72.08,385.70) rectangle (667.14,385.70);
\end{scope}
\begin{scope}
\path[clip] (  0.00,  0.00) rectangle (722.70,433.62);
\end{scope}
\begin{scope}
\path[clip] ( 72.08, 55.41) rectangle (667.14,385.70);
\definecolor[named]{fillColor}{rgb}{0.90,0.90,0.90}

\draw[fill=fillColor,draw opacity=0.00,] ( 72.08, 55.41) rectangle (667.14,385.70);
\definecolor[named]{drawColor}{rgb}{0.95,0.95,0.95}

\draw[color=drawColor,line width= 0.3pt,line cap=round,line join=round,fill opacity=0.00,] ( 72.08, 64.97) --
	(667.14, 64.97);

\draw[color=drawColor,line width= 0.3pt,line cap=round,line join=round,fill opacity=0.00,] ( 72.08, 84.86) --
	(667.14, 84.86);

\draw[color=drawColor,line width= 0.3pt,line cap=round,line join=round,fill opacity=0.00,] ( 72.08,104.75) --
	(667.14,104.75);

\draw[color=drawColor,line width= 0.3pt,line cap=round,line join=round,fill opacity=0.00,] ( 72.08,124.64) --
	(667.14,124.64);

\draw[color=drawColor,line width= 0.3pt,line cap=round,line join=round,fill opacity=0.00,] ( 72.08,144.52) --
	(667.14,144.52);

\draw[color=drawColor,line width= 0.3pt,line cap=round,line join=round,fill opacity=0.00,] ( 72.08,164.41) --
	(667.14,164.41);

\draw[color=drawColor,line width= 0.3pt,line cap=round,line join=round,fill opacity=0.00,] ( 72.08,184.30) --
	(667.14,184.30);

\draw[color=drawColor,line width= 0.3pt,line cap=round,line join=round,fill opacity=0.00,] ( 72.08,204.19) --
	(667.14,204.19);

\draw[color=drawColor,line width= 0.3pt,line cap=round,line join=round,fill opacity=0.00,] ( 72.08,224.08) --
	(667.14,224.08);

\draw[color=drawColor,line width= 0.3pt,line cap=round,line join=round,fill opacity=0.00,] ( 72.08,243.96) --
	(667.14,243.96);

\draw[color=drawColor,line width= 0.3pt,line cap=round,line join=round,fill opacity=0.00,] ( 72.08,263.85) --
	(667.14,263.85);

\draw[color=drawColor,line width= 0.3pt,line cap=round,line join=round,fill opacity=0.00,] ( 72.08,283.74) --
	(667.14,283.74);

\draw[color=drawColor,line width= 0.3pt,line cap=round,line join=round,fill opacity=0.00,] ( 72.08,303.63) --
	(667.14,303.63);

\draw[color=drawColor,line width= 0.3pt,line cap=round,line join=round,fill opacity=0.00,] ( 72.08,323.51) --
	(667.14,323.51);

\draw[color=drawColor,line width= 0.3pt,line cap=round,line join=round,fill opacity=0.00,] ( 72.08,343.40) --
	(667.14,343.40);

\draw[color=drawColor,line width= 0.3pt,line cap=round,line join=round,fill opacity=0.00,] ( 72.08,363.29) --
	(667.14,363.29);

\draw[color=drawColor,line width= 0.3pt,line cap=round,line join=round,fill opacity=0.00,] ( 72.08,383.18) --
	(667.14,383.18);

\draw[color=drawColor,line width= 0.3pt,line cap=round,line join=round,fill opacity=0.00,] (118.84, 55.41) --
	(118.84,385.70);

\draw[color=drawColor,line width= 0.3pt,line cap=round,line join=round,fill opacity=0.00,] (168.56, 55.41) --
	(168.56,385.70);

\draw[color=drawColor,line width= 0.3pt,line cap=round,line join=round,fill opacity=0.00,] (218.28, 55.41) --
	(218.28,385.70);

\draw[color=drawColor,line width= 0.3pt,line cap=round,line join=round,fill opacity=0.00,] (268.00, 55.41) --
	(268.00,385.70);

\draw[color=drawColor,line width= 0.3pt,line cap=round,line join=round,fill opacity=0.00,] (317.72, 55.41) --
	(317.72,385.70);

\draw[color=drawColor,line width= 0.3pt,line cap=round,line join=round,fill opacity=0.00,] (367.44, 55.41) --
	(367.44,385.70);

\draw[color=drawColor,line width= 0.3pt,line cap=round,line join=round,fill opacity=0.00,] (417.16, 55.41) --
	(417.16,385.70);

\draw[color=drawColor,line width= 0.3pt,line cap=round,line join=round,fill opacity=0.00,] (466.88, 55.41) --
	(466.88,385.70);

\draw[color=drawColor,line width= 0.3pt,line cap=round,line join=round,fill opacity=0.00,] (516.60, 55.41) --
	(516.60,385.70);

\draw[color=drawColor,line width= 0.3pt,line cap=round,line join=round,fill opacity=0.00,] (566.32, 55.41) --
	(566.32,385.70);

\draw[color=drawColor,line width= 0.3pt,line cap=round,line join=round,fill opacity=0.00,] (616.03, 55.41) --
	(616.03,385.70);

\draw[color=drawColor,line width= 0.3pt,line cap=round,line join=round,fill opacity=0.00,] (665.75, 55.41) --
	(665.75,385.70);
\definecolor[named]{drawColor}{rgb}{1.00,1.00,1.00}

\draw[color=drawColor,line width= 0.6pt,line cap=round,line join=round,fill opacity=0.00,] ( 72.08, 84.86) --
	(667.14, 84.86);

\draw[color=drawColor,line width= 0.6pt,line cap=round,line join=round,fill opacity=0.00,] ( 72.08,124.64) --
	(667.14,124.64);

\draw[color=drawColor,line width= 0.6pt,line cap=round,line join=round,fill opacity=0.00,] ( 72.08,164.41) --
	(667.14,164.41);

\draw[color=drawColor,line width= 0.6pt,line cap=round,line join=round,fill opacity=0.00,] ( 72.08,204.19) --
	(667.14,204.19);

\draw[color=drawColor,line width= 0.6pt,line cap=round,line join=round,fill opacity=0.00,] ( 72.08,243.96) --
	(667.14,243.96);

\draw[color=drawColor,line width= 0.6pt,line cap=round,line join=round,fill opacity=0.00,] ( 72.08,283.74) --
	(667.14,283.74);

\draw[color=drawColor,line width= 0.6pt,line cap=round,line join=round,fill opacity=0.00,] ( 72.08,323.51) --
	(667.14,323.51);

\draw[color=drawColor,line width= 0.6pt,line cap=round,line join=round,fill opacity=0.00,] ( 72.08,363.29) --
	(667.14,363.29);

\draw[color=drawColor,line width= 0.6pt,line cap=round,line join=round,fill opacity=0.00,] (168.56, 55.41) --
	(168.56,385.70);

\draw[color=drawColor,line width= 0.6pt,line cap=round,line join=round,fill opacity=0.00,] (268.00, 55.41) --
	(268.00,385.70);

\draw[color=drawColor,line width= 0.6pt,line cap=round,line join=round,fill opacity=0.00,] (367.44, 55.41) --
	(367.44,385.70);

\draw[color=drawColor,line width= 0.6pt,line cap=round,line join=round,fill opacity=0.00,] (466.88, 55.41) --
	(466.88,385.70);

\draw[color=drawColor,line width= 0.6pt,line cap=round,line join=round,fill opacity=0.00,] (566.32, 55.41) --
	(566.32,385.70);
\definecolor[named]{drawColor}{rgb}{0.00,0.00,0.00}

\draw[color=drawColor,line width= 0.6pt,line join=round,fill opacity=0.00,] (168.93,139.04) --
	(171.72,158.54) --
	(172.14,160.22) --
	(173.75,165.06) --
	(194.87,186.18) --
	(202.38,190.69) --
	(207.82,195.04) --
	(212.02,196.72) --
	(230.20,200.35) --
	(362.69,200.35);

\draw[color=drawColor,line width= 0.6pt,line join=round,fill opacity=0.00,] ( 99.13, 70.42) --
	(106.85, 78.14) --
	(194.87,177.16) --
	(202.38,190.69) --
	(207.82,195.04) --
	(212.02,196.72) --
	(230.20,200.35) --
	(362.69,200.35);

\draw[color=drawColor,line width= 0.6pt,line join=round,fill opacity=0.00,] (100.85,148.46) --
	(101.22,148.62) --
	(106.85,151.83) --
	(114.19,155.50) --
	(122.28,158.54) --
	(125.64,160.22) --
	(194.87,186.18) --
	(202.38,190.69) --
	(207.82,195.04) --
	(212.02,196.72) --
	(230.20,200.35) --
	(362.69,200.35);

\draw[color=drawColor,line width= 0.6pt,line join=round,fill opacity=0.00,] (158.13,163.43) --
	(167.75,160.22) --
	(169.10,161.57) --
	(173.75,165.06) --
	(194.87,186.18) --
	(202.38,190.69) --
	(207.82,195.04) --
	(212.02,196.72) --
	(230.20,200.35) --
	(362.69,200.35);

\draw[color=drawColor,line width= 0.6pt,line join=round,fill opacity=0.00,] (150.81,153.73) --
	(159.69,155.50) --
	(164.74,158.54) --
	(166.84,160.22) --
	(169.10,161.57) --
	(173.75,165.06) --
	(194.87,186.18) --
	(202.38,190.69) --
	(207.82,195.04) --
	(212.02,196.72) --
	(230.20,200.35) --
	(362.69,200.35);

\draw[color=drawColor,line width= 0.6pt,line join=round,fill opacity=0.00,] (620.70,262.27) --
	(595.18,236.74) --
	(564.95,215.15) --
	(558.82,211.65) --
	(533.90,195.04) --
	(528.87,196.72) --
	(507.06,200.35) --
	(435.77,200.35) --
	(362.69,200.35);

\draw[color=drawColor,line width= 0.6pt,line join=round,fill opacity=0.00,] (640.09,370.68) --
	(603.35,333.94) --
	(595.18,324.74) --
	(558.82,278.00) --
	(507.06,200.35) --
	(435.77,200.35) --
	(362.69,200.35);

\draw[color=drawColor,line width= 0.6pt,line join=round,fill opacity=0.00,] (631.93,303.74) --
	(603.35,275.17) --
	(595.18,268.01) --
	(558.82,231.66) --
	(528.87,196.72) --
	(507.06,200.35) --
	(435.77,200.35) --
	(362.69,200.35);

\draw[color=drawColor,line width= 0.6pt,line join=round,fill opacity=0.00,] (589.71,243.42) --
	(561.45,215.15) --
	(558.82,211.65) --
	(533.90,195.04) --
	(528.87,196.72) --
	(507.06,200.35) --
	(435.77,200.35) --
	(362.69,200.35);

\draw[color=drawColor,line width= 0.6pt,line join=round,fill opacity=0.00,] (466.57,148.35) --
	(466.52,148.62) --
	(464.80,155.50) --
	(463.79,158.54) --
	(463.37,160.22) --
	(453.21,190.69) --
	(452.12,195.04) --
	(451.28,196.72) --
	(447.65,200.35) --
	(435.77,200.35) --
	(362.69,200.35);
\definecolor[named]{fillColor}{rgb}{0.00,0.00,0.00}

\draw[fill=fillColor,draw opacity=0.00,] (466.57,148.35) circle (  2.13);

\draw[color=drawColor,line cap=round,line join=round,fill opacity=0.00,] (168.93,139.04) circle (  2.13);

\draw[color=drawColor,line cap=round,line join=round,fill opacity=0.00,] ( 99.13, 70.42) circle (  2.13);

\draw[color=drawColor,line cap=round,line join=round,fill opacity=0.00,] (100.85,148.46) circle (  2.13);

\draw[color=drawColor,line cap=round,line join=round,fill opacity=0.00,] (158.13,163.43) circle (  2.13);

\draw[color=drawColor,line cap=round,line join=round,fill opacity=0.00,] (150.81,153.73) circle (  2.13);

\draw[color=drawColor,line cap=round,line join=round,fill opacity=0.00,] (620.70,262.27) circle (  2.13);

\draw[color=drawColor,line cap=round,line join=round,fill opacity=0.00,] (640.09,370.68) circle (  2.13);

\draw[color=drawColor,line cap=round,line join=round,fill opacity=0.00,] (631.93,303.74) circle (  2.13);

\draw[color=drawColor,line cap=round,line join=round,fill opacity=0.00,] (589.71,243.42) circle (  2.13);

\draw[color=drawColor,line cap=round,line join=round,fill opacity=0.00,] (466.57,148.35) circle (  2.13);
\definecolor[named]{fillColor}{rgb}{1.00,0.00,0.00}

\draw[fill=fillColor,draw opacity=0.00,] (314.56,200.35) circle (  2.13);

\draw[fill=fillColor,draw opacity=0.00,] (314.56,200.35) circle (  2.13);

\draw[fill=fillColor,draw opacity=0.00,] (314.56,200.35) circle (  2.13);

\draw[fill=fillColor,draw opacity=0.00,] (314.56,200.35) circle (  2.13);

\draw[fill=fillColor,draw opacity=0.00,] (314.56,200.35) circle (  2.13);

\draw[fill=fillColor,draw opacity=0.00,] (410.81,200.35) circle (  2.13);

\draw[fill=fillColor,draw opacity=0.00,] (410.81,200.35) circle (  2.13);

\draw[fill=fillColor,draw opacity=0.00,] (410.81,200.35) circle (  2.13);

\draw[fill=fillColor,draw opacity=0.00,] (410.81,200.35) circle (  2.13);

\draw[fill=fillColor,draw opacity=0.00,] (410.81,200.35) circle (  2.13);

\node[color=drawColor,anchor=base,inner sep=0pt, outer sep=0pt, scale=  0.59] at (407.21,122.39) {Joins the left cluster on $\alpha^1$\ before joining right cluster.%
};
\definecolor[named]{drawColor}{rgb}{1.00,0.00,0.00}

\node[color=drawColor,anchor=base,inner sep=0pt, outer sep=0pt, scale=  0.59] at (327.66,221.83) {Solution at $\lambda=0.18$\ yields 2 clusters.%
};
\end{scope}
\begin{scope}
\path[clip] (  0.00,  0.00) rectangle (722.70,433.62);
\end{scope}
\begin{scope}
\path[clip] ( 72.08, 55.41) rectangle (667.14, 55.41);
\end{scope}
\begin{scope}
\path[clip] (  0.00,  0.00) rectangle (722.70,433.62);
\end{scope}
\begin{scope}
\path[clip] (  0.00,  0.00) rectangle (722.70,433.62);
\definecolor[named]{drawColor}{rgb}{0.00,0.00,0.00}

\node[color=drawColor,anchor=base,inner sep=0pt, outer sep=0pt, scale=  0.42] at (168.56, 32.08) {-0.5%
};

\node[color=drawColor,anchor=base,inner sep=0pt, outer sep=0pt, scale=  0.42] at (268.00, 32.08) {0.0%
};

\node[color=drawColor,anchor=base,inner sep=0pt, outer sep=0pt, scale=  0.42] at (367.44, 32.08) {0.5%
};

\node[color=drawColor,anchor=base,inner sep=0pt, outer sep=0pt, scale=  0.42] at (466.88, 32.08) {1.0%
};

\node[color=drawColor,anchor=base,inner sep=0pt, outer sep=0pt, scale=  0.42] at (566.32, 32.08) {1.5%
};
\end{scope}
\begin{scope}
\path[clip] (  0.00,  0.00) rectangle (722.70,433.62);
\definecolor[named]{drawColor}{rgb}{0.50,0.50,0.50}

\draw[color=drawColor,line width= 0.6pt,line cap=round,line join=round,fill opacity=0.00,] (168.56, 51.14) -- (168.56, 55.41);

\draw[color=drawColor,line width= 0.6pt,line cap=round,line join=round,fill opacity=0.00,] (268.00, 51.14) -- (268.00, 55.41);

\draw[color=drawColor,line width= 0.6pt,line cap=round,line join=round,fill opacity=0.00,] (367.44, 51.14) -- (367.44, 55.41);

\draw[color=drawColor,line width= 0.6pt,line cap=round,line join=round,fill opacity=0.00,] (466.88, 51.14) -- (466.88, 55.41);

\draw[color=drawColor,line width= 0.6pt,line cap=round,line join=round,fill opacity=0.00,] (566.32, 51.14) -- (566.32, 55.41);
\end{scope}
\begin{scope}
\path[clip] (  0.00,  0.00) rectangle (722.70,433.62);
\end{scope}
\begin{scope}
\path[clip] (  0.00,  0.00) rectangle (722.70,433.62);
\end{scope}
\begin{scope}
\path[clip] (  0.00,  0.00) rectangle (722.70,433.62);
\end{scope}
\begin{scope}
\path[clip] ( 72.08, 32.08) rectangle (667.14, 32.08);
\end{scope}
\begin{scope}
\path[clip] (  0.00,  0.00) rectangle (722.70,433.62);
\end{scope}
\begin{scope}
\path[clip] (667.14,385.70) rectangle (667.14,385.70);
\end{scope}
\begin{scope}
\path[clip] (  0.00,  0.00) rectangle (722.70,433.62);
\end{scope}
\begin{scope}
\path[clip] (667.14,385.70) rectangle (667.14,385.70);
\end{scope}
\begin{scope}
\path[clip] (  0.00,  0.00) rectangle (722.70,433.62);
\end{scope}
\begin{scope}
\path[clip] (667.14, 55.41) rectangle (667.14,385.70);
\end{scope}
\begin{scope}
\path[clip] (  0.00,  0.00) rectangle (722.70,433.62);
\end{scope}
\begin{scope}
\path[clip] (667.14, 55.41) rectangle (667.14, 55.41);
\end{scope}
\begin{scope}
\path[clip] (  0.00,  0.00) rectangle (722.70,433.62);
\end{scope}
\begin{scope}
\path[clip] (667.14, 32.08) rectangle (667.14, 55.41);
\end{scope}
\begin{scope}
\path[clip] (  0.00,  0.00) rectangle (722.70,433.62);
\end{scope}
\begin{scope}
\path[clip] (667.14, 32.08) rectangle (667.14, 32.08);
\end{scope}
\begin{scope}
\path[clip] (  0.00,  0.00) rectangle (722.70,433.62);
\end{scope}
\begin{scope}
\path[clip] (667.14,385.70) rectangle (667.14,385.70);
\end{scope}
\begin{scope}
\path[clip] (  0.00,  0.00) rectangle (722.70,433.62);
\end{scope}
\begin{scope}
\path[clip] (667.14,385.70) rectangle (667.14,385.70);
\end{scope}
\begin{scope}
\path[clip] (  0.00,  0.00) rectangle (722.70,433.62);
\end{scope}
\begin{scope}
\path[clip] (667.14, 55.41) rectangle (667.14,385.70);
\end{scope}
\begin{scope}
\path[clip] (  0.00,  0.00) rectangle (722.70,433.62);
\end{scope}
\begin{scope}
\path[clip] (667.14, 55.41) rectangle (667.14, 55.41);
\end{scope}
\begin{scope}
\path[clip] (  0.00,  0.00) rectangle (722.70,433.62);
\end{scope}
\begin{scope}
\path[clip] (667.14, 32.08) rectangle (667.14, 55.41);
\end{scope}
\begin{scope}
\path[clip] (  0.00,  0.00) rectangle (722.70,433.62);
\end{scope}
\begin{scope}
\path[clip] (667.14, 32.08) rectangle (667.14, 32.08);
\end{scope}
\begin{scope}
\path[clip] (  0.00,  0.00) rectangle (722.70,433.62);
\end{scope}
\begin{scope}
\path[clip] (667.14,385.70) rectangle (667.14,385.70);
\end{scope}
\begin{scope}
\path[clip] (  0.00,  0.00) rectangle (722.70,433.62);
\end{scope}
\begin{scope}
\path[clip] (667.14,385.70) rectangle (667.14,385.70);
\end{scope}
\begin{scope}
\path[clip] (  0.00,  0.00) rectangle (722.70,433.62);
\end{scope}
\begin{scope}
\path[clip] (667.14, 55.41) rectangle (667.14,385.70);
\end{scope}
\begin{scope}
\path[clip] (  0.00,  0.00) rectangle (722.70,433.62);
\end{scope}
\begin{scope}
\path[clip] (667.14, 55.41) rectangle (667.14, 55.41);
\end{scope}
\begin{scope}
\path[clip] (  0.00,  0.00) rectangle (722.70,433.62);
\end{scope}
\begin{scope}
\path[clip] (667.14, 32.08) rectangle (667.14, 55.41);
\end{scope}
\begin{scope}
\path[clip] (  0.00,  0.00) rectangle (722.70,433.62);
\end{scope}
\begin{scope}
\path[clip] (667.14, 32.08) rectangle (667.14, 32.08);
\end{scope}
\begin{scope}
\path[clip] (  0.00,  0.00) rectangle (722.70,433.62);
\end{scope}
\begin{scope}
\path[clip] (  0.00,  0.00) rectangle (722.70,433.62);
\end{scope}
\begin{scope}
\path[clip] (  0.00,  0.00) rectangle (722.70,433.62);
\end{scope}
\end{tikzpicture}
% Created by tikzDevice version 0.6.1 on 2011-06-26 17:52:49
% !TEX encoding = UTF-8 Unicode
\begin{tikzpicture}[x=1pt,y=1pt]
\definecolor[named]{drawColor}{rgb}{0.00,0.00,0.00}
\definecolor[named]{fillColor}{rgb}{1.00,1.00,1.00}
\fill[color=fillColor,] (0,0) rectangle (289.08,433.62);
\begin{scope}
\path[clip] (  0.00,  0.00) rectangle (289.08,433.62);
\end{scope}
\begin{scope}
\path[clip] (  0.00,  0.00) rectangle (289.08,433.62);
\end{scope}
\begin{scope}
\path[clip] (  0.00,  0.00) rectangle (289.08,433.62);
\end{scope}
\begin{scope}
\path[clip] (  0.00,  0.00) rectangle (289.08,433.62);
\end{scope}
\begin{scope}
\path[clip] (  0.00,  0.00) rectangle (289.08,433.62);
\end{scope}
\begin{scope}
\path[clip] (  0.00,  0.00) rectangle (289.08,433.62);
\end{scope}
\begin{scope}
\path[clip] (  0.00,  0.00) rectangle (289.08,433.62);
\end{scope}
\begin{scope}
\path[clip] (  0.00,  0.00) rectangle (289.08,433.62);
\end{scope}
\begin{scope}
\path[clip] (  0.00,  0.00) rectangle (289.08,433.62);
\end{scope}
\begin{scope}
\path[clip] (  0.00,  0.00) rectangle (289.08,433.62);
\definecolor[named]{fillColor}{rgb}{1.00,1.00,1.00}

\draw[fill=fillColor,draw opacity=0.00,] (  0.00,  0.00) rectangle (289.08,433.62);
\end{scope}
\begin{scope}
\path[clip] (  0.00,  0.00) rectangle (289.08,433.62);
\end{scope}
\begin{scope}
\path[clip] (  0.00,  0.00) rectangle (289.08,433.62);
\definecolor[named]{drawColor}{rgb}{0.00,0.00,0.00}

\node[color=drawColor,anchor=base,inner sep=0pt, outer sep=0pt, scale=  0.42] at (138.90,  7.23) {Location in the regularization path $\lambda$%
};
\end{scope}
\begin{scope}
\path[clip] (  0.00,  0.00) rectangle (289.08,433.62);
\definecolor[named]{drawColor}{rgb}{0.00,0.00,0.00}

\node[rotate= 90.00,color=drawColor,anchor=base,inner sep=0pt, outer sep=0pt, scale=  0.42] at ( 12.37,218.39) {Optimal value of $\ell_1$ clusterpath%
};
\end{scope}
\begin{scope}
\path[clip] (  0.00,  0.00) rectangle (289.08,433.62);
\end{scope}
\begin{scope}
\path[clip] ( 32.08,404.71) rectangle ( 59.25,404.71);
\end{scope}
\begin{scope}
\path[clip] (  0.00,  0.00) rectangle (289.08,433.62);
\end{scope}
\begin{scope}
\path[clip] ( 32.08,404.71) rectangle ( 59.25,404.71);
\end{scope}
\begin{scope}
\path[clip] (  0.00,  0.00) rectangle (289.08,433.62);
\end{scope}
\begin{scope}
\path[clip] (  0.00,  0.00) rectangle (289.08,433.62);
\end{scope}
\begin{scope}
\path[clip] (  0.00,  0.00) rectangle (289.08,433.62);
\end{scope}
\begin{scope}
\path[clip] ( 32.08,226.45) rectangle ( 59.25,233.68);
\end{scope}
\begin{scope}
\path[clip] (  0.00,  0.00) rectangle (289.08,433.62);
\end{scope}
\begin{scope}
\path[clip] (  0.00,  0.00) rectangle (289.08,433.62);
\end{scope}
\begin{scope}
\path[clip] (  0.00,  0.00) rectangle (289.08,433.62);
\end{scope}
\begin{scope}
\path[clip] ( 32.08, 55.41) rectangle ( 59.25, 55.41);
\end{scope}
\begin{scope}
\path[clip] (  0.00,  0.00) rectangle (289.08,433.62);
\end{scope}
\begin{scope}
\path[clip] ( 32.08, 32.08) rectangle ( 59.25, 55.41);
\end{scope}
\begin{scope}
\path[clip] (  0.00,  0.00) rectangle (289.08,433.62);
\end{scope}
\begin{scope}
\path[clip] ( 32.08, 32.08) rectangle ( 59.25, 32.08);
\end{scope}
\begin{scope}
\path[clip] (  0.00,  0.00) rectangle (289.08,433.62);
\end{scope}
\begin{scope}
\path[clip] ( 59.25,404.71) rectangle ( 59.25,404.71);
\end{scope}
\begin{scope}
\path[clip] (  0.00,  0.00) rectangle (289.08,433.62);
\end{scope}
\begin{scope}
\path[clip] ( 59.25,404.71) rectangle ( 59.25,404.71);
\end{scope}
\begin{scope}
\path[clip] (  0.00,  0.00) rectangle (289.08,433.62);
\end{scope}
\begin{scope}
\path[clip] ( 59.25,233.68) rectangle ( 59.25,404.71);
\end{scope}
\begin{scope}
\path[clip] (  0.00,  0.00) rectangle (289.08,433.62);
\end{scope}
\begin{scope}
\path[clip] ( 59.25,226.45) rectangle ( 59.25,233.68);
\end{scope}
\begin{scope}
\path[clip] (  0.00,  0.00) rectangle (289.08,433.62);
\end{scope}
\begin{scope}
\path[clip] ( 59.25, 55.41) rectangle ( 59.25,226.45);
\end{scope}
\begin{scope}
\path[clip] (  0.00,  0.00) rectangle (289.08,433.62);
\end{scope}
\begin{scope}
\path[clip] ( 59.25, 55.41) rectangle ( 59.25, 55.41);
\end{scope}
\begin{scope}
\path[clip] (  0.00,  0.00) rectangle (289.08,433.62);
\end{scope}
\begin{scope}
\path[clip] ( 59.25, 32.08) rectangle ( 59.25, 55.41);
\end{scope}
\begin{scope}
\path[clip] (  0.00,  0.00) rectangle (289.08,433.62);
\end{scope}
\begin{scope}
\path[clip] ( 59.25, 32.08) rectangle ( 59.25, 32.08);
\end{scope}
\begin{scope}
\path[clip] (  0.00,  0.00) rectangle (289.08,433.62);
\end{scope}
\begin{scope}
\path[clip] ( 59.25,404.71) rectangle (228.22,404.71);
\end{scope}
\begin{scope}
\path[clip] (  0.00,  0.00) rectangle (289.08,433.62);
\end{scope}
\begin{scope}
\path[clip] ( 59.25,404.71) rectangle (228.22,404.71);
\end{scope}
\begin{scope}
\path[clip] (  0.00,  0.00) rectangle (289.08,433.62);
\end{scope}
\begin{scope}
\path[clip] ( 59.25,233.68) rectangle (228.22,404.71);
\end{scope}
\begin{scope}
\path[clip] (  0.00,  0.00) rectangle (289.08,433.62);
\end{scope}
\begin{scope}
\path[clip] ( 59.25,226.45) rectangle (228.22,233.68);
\end{scope}
\begin{scope}
\path[clip] (  0.00,  0.00) rectangle (289.08,433.62);
\end{scope}
\begin{scope}
\path[clip] ( 59.25, 55.41) rectangle (228.22,226.45);
\end{scope}
\begin{scope}
\path[clip] (  0.00,  0.00) rectangle (289.08,433.62);
\end{scope}
\begin{scope}
\path[clip] ( 59.25, 55.41) rectangle (228.22, 55.41);
\end{scope}
\begin{scope}
\path[clip] (  0.00,  0.00) rectangle (289.08,433.62);
\end{scope}
\begin{scope}
\path[clip] (  0.00,  0.00) rectangle (289.08,433.62);
\end{scope}
\begin{scope}
\path[clip] (  0.00,  0.00) rectangle (289.08,433.62);
\end{scope}
\begin{scope}
\path[clip] ( 59.25, 32.08) rectangle (228.22, 32.08);
\end{scope}
\begin{scope}
\path[clip] (  0.00,  0.00) rectangle (289.08,433.62);
\end{scope}
\begin{scope}
\path[clip] (228.22,404.71) rectangle (228.22,404.71);
\end{scope}
\begin{scope}
\path[clip] (  0.00,  0.00) rectangle (289.08,433.62);
\end{scope}
\begin{scope}
\path[clip] (228.22,404.71) rectangle (228.22,404.71);
\end{scope}
\begin{scope}
\path[clip] (  0.00,  0.00) rectangle (289.08,433.62);
\end{scope}
\begin{scope}
\path[clip] (228.22,233.68) rectangle (228.22,404.71);
\end{scope}
\begin{scope}
\path[clip] (  0.00,  0.00) rectangle (289.08,433.62);
\end{scope}
\begin{scope}
\path[clip] (228.22,226.45) rectangle (228.22,233.68);
\end{scope}
\begin{scope}
\path[clip] (  0.00,  0.00) rectangle (289.08,433.62);
\end{scope}
\begin{scope}
\path[clip] (228.22, 55.41) rectangle (228.22,226.45);
\end{scope}
\begin{scope}
\path[clip] (  0.00,  0.00) rectangle (289.08,433.62);
\end{scope}
\begin{scope}
\path[clip] (228.22, 55.41) rectangle (228.22, 55.41);
\end{scope}
\begin{scope}
\path[clip] (  0.00,  0.00) rectangle (289.08,433.62);
\end{scope}
\begin{scope}
\path[clip] (228.22, 32.08) rectangle (228.22, 55.41);
\end{scope}
\begin{scope}
\path[clip] (  0.00,  0.00) rectangle (289.08,433.62);
\end{scope}
\begin{scope}
\path[clip] (228.22, 32.08) rectangle (228.22, 32.08);
\end{scope}
\begin{scope}
\path[clip] (  0.00,  0.00) rectangle (289.08,433.62);
\end{scope}
\begin{scope}
\path[clip] (228.22,404.71) rectangle (245.72,404.71);
\end{scope}
\begin{scope}
\path[clip] (  0.00,  0.00) rectangle (289.08,433.62);
\end{scope}
\begin{scope}
\path[clip] (228.22,404.71) rectangle (245.72,404.71);
\end{scope}
\begin{scope}
\path[clip] (  0.00,  0.00) rectangle (289.08,433.62);
\end{scope}
\begin{scope}
\path[clip] (228.22,233.68) rectangle (245.72,404.71);
\end{scope}
\begin{scope}
\path[clip] (  0.00,  0.00) rectangle (289.08,433.62);
\end{scope}
\begin{scope}
\path[clip] (228.22,226.45) rectangle (245.72,233.68);
\end{scope}
\begin{scope}
\path[clip] (  0.00,  0.00) rectangle (289.08,433.62);
\end{scope}
\begin{scope}
\path[clip] (228.22, 55.41) rectangle (245.72,226.45);
\end{scope}
\begin{scope}
\path[clip] (  0.00,  0.00) rectangle (289.08,433.62);
\end{scope}
\begin{scope}
\path[clip] (228.22, 55.41) rectangle (245.72, 55.41);
\end{scope}
\begin{scope}
\path[clip] (  0.00,  0.00) rectangle (289.08,433.62);
\end{scope}
\begin{scope}
\path[clip] (228.22, 32.08) rectangle (245.72, 55.41);
\end{scope}
\begin{scope}
\path[clip] (  0.00,  0.00) rectangle (289.08,433.62);
\end{scope}
\begin{scope}
\path[clip] (228.22, 32.08) rectangle (245.72, 32.08);
\end{scope}
\begin{scope}
\path[clip] (  0.00,  0.00) rectangle (289.08,433.62);
\end{scope}
\begin{scope}
\path[clip] (245.72,404.71) rectangle (245.72,404.71);
\end{scope}
\begin{scope}
\path[clip] (  0.00,  0.00) rectangle (289.08,433.62);
\end{scope}
\begin{scope}
\path[clip] (245.72,404.71) rectangle (245.72,404.71);
\end{scope}
\begin{scope}
\path[clip] (  0.00,  0.00) rectangle (289.08,433.62);
\end{scope}
\begin{scope}
\path[clip] (245.72,233.68) rectangle (245.72,404.71);
\end{scope}
\begin{scope}
\path[clip] (  0.00,  0.00) rectangle (289.08,433.62);
\end{scope}
\begin{scope}
\path[clip] (245.72,226.45) rectangle (245.72,233.68);
\end{scope}
\begin{scope}
\path[clip] (  0.00,  0.00) rectangle (289.08,433.62);
\end{scope}
\begin{scope}
\path[clip] (245.72, 55.41) rectangle (245.72,226.45);
\end{scope}
\begin{scope}
\path[clip] (  0.00,  0.00) rectangle (289.08,433.62);
\end{scope}
\begin{scope}
\path[clip] (245.72, 55.41) rectangle (245.72, 55.41);
\end{scope}
\begin{scope}
\path[clip] (  0.00,  0.00) rectangle (289.08,433.62);
\end{scope}
\begin{scope}
\path[clip] (245.72, 32.08) rectangle (245.72, 55.41);
\end{scope}
\begin{scope}
\path[clip] (  0.00,  0.00) rectangle (289.08,433.62);
\end{scope}
\begin{scope}
\path[clip] (245.72, 32.08) rectangle (245.72, 32.08);
\end{scope}
\begin{scope}
\path[clip] (  0.00,  0.00) rectangle (289.08,433.62);
\end{scope}
\begin{scope}
\path[clip] ( 32.08,404.71) rectangle ( 59.25,404.71);
\end{scope}
\begin{scope}
\path[clip] (  0.00,  0.00) rectangle (289.08,433.62);
\end{scope}
\begin{scope}
\path[clip] ( 32.08,404.71) rectangle ( 59.25,404.71);
\end{scope}
\begin{scope}
\path[clip] (  0.00,  0.00) rectangle (289.08,433.62);
\end{scope}
\begin{scope}
\path[clip] (  0.00,  0.00) rectangle (289.08,433.62);
\definecolor[named]{drawColor}{rgb}{0.50,0.50,0.50}

\node[color=drawColor,anchor=base east,inner sep=0pt, outer sep=0pt, scale=  0.40] at ( 39.09,247.41) {-0.8%
};

\node[color=drawColor,anchor=base east,inner sep=0pt, outer sep=0pt, scale=  0.40] at ( 39.09,268.00) {-0.6%
};

\node[color=drawColor,anchor=base east,inner sep=0pt, outer sep=0pt, scale=  0.40] at ( 39.09,288.60) {-0.4%
};

\node[color=drawColor,anchor=base east,inner sep=0pt, outer sep=0pt, scale=  0.40] at ( 39.09,309.20) {-0.2%
};

\node[color=drawColor,anchor=base east,inner sep=0pt, outer sep=0pt, scale=  0.40] at ( 39.09,329.80) {0.0%
};

\node[color=drawColor,anchor=base east,inner sep=0pt, outer sep=0pt, scale=  0.40] at ( 39.09,350.39) {0.2%
};

\node[color=drawColor,anchor=base east,inner sep=0pt, outer sep=0pt, scale=  0.40] at ( 39.09,370.99) {0.4%
};

\node[color=drawColor,anchor=base east,inner sep=0pt, outer sep=0pt, scale=  0.40] at ( 39.09,391.59) {0.6%
};
\end{scope}
\begin{scope}
\path[clip] (  0.00,  0.00) rectangle (289.08,433.62);
\definecolor[named]{drawColor}{rgb}{0.50,0.50,0.50}

\draw[color=drawColor,line width= 0.6pt,line cap=round,line join=round,fill opacity=0.00,] ( 54.99,248.93) -- ( 59.25,248.93);

\draw[color=drawColor,line width= 0.6pt,line cap=round,line join=round,fill opacity=0.00,] ( 54.99,269.52) -- ( 59.25,269.52);

\draw[color=drawColor,line width= 0.6pt,line cap=round,line join=round,fill opacity=0.00,] ( 54.99,290.12) -- ( 59.25,290.12);

\draw[color=drawColor,line width= 0.6pt,line cap=round,line join=round,fill opacity=0.00,] ( 54.99,310.72) -- ( 59.25,310.72);

\draw[color=drawColor,line width= 0.6pt,line cap=round,line join=round,fill opacity=0.00,] ( 54.99,331.32) -- ( 59.25,331.32);

\draw[color=drawColor,line width= 0.6pt,line cap=round,line join=round,fill opacity=0.00,] ( 54.99,351.91) -- ( 59.25,351.91);

\draw[color=drawColor,line width= 0.6pt,line cap=round,line join=round,fill opacity=0.00,] ( 54.99,372.51) -- ( 59.25,372.51);

\draw[color=drawColor,line width= 0.6pt,line cap=round,line join=round,fill opacity=0.00,] ( 54.99,393.11) -- ( 59.25,393.11);
\end{scope}
\begin{scope}
\path[clip] (  0.00,  0.00) rectangle (289.08,433.62);
\end{scope}
\begin{scope}
\path[clip] (  0.00,  0.00) rectangle (289.08,433.62);
\end{scope}
\begin{scope}
\path[clip] (  0.00,  0.00) rectangle (289.08,433.62);
\end{scope}
\begin{scope}
\path[clip] ( 32.08,226.45) rectangle ( 59.25,233.68);
\end{scope}
\begin{scope}
\path[clip] (  0.00,  0.00) rectangle (289.08,433.62);
\end{scope}
\begin{scope}
\path[clip] (  0.00,  0.00) rectangle (289.08,433.62);
\definecolor[named]{drawColor}{rgb}{0.50,0.50,0.50}

\node[color=drawColor,anchor=base east,inner sep=0pt, outer sep=0pt, scale=  0.40] at ( 39.09, 81.62) {-0.5%
};

\node[color=drawColor,anchor=base east,inner sep=0pt, outer sep=0pt, scale=  0.40] at ( 39.09,110.20) {0.0%
};

\node[color=drawColor,anchor=base east,inner sep=0pt, outer sep=0pt, scale=  0.40] at ( 39.09,138.78) {0.5%
};

\node[color=drawColor,anchor=base east,inner sep=0pt, outer sep=0pt, scale=  0.40] at ( 39.09,167.37) {1.0%
};

\node[color=drawColor,anchor=base east,inner sep=0pt, outer sep=0pt, scale=  0.40] at ( 39.09,195.95) {1.5%
};
\end{scope}
\begin{scope}
\path[clip] (  0.00,  0.00) rectangle (289.08,433.62);
\definecolor[named]{drawColor}{rgb}{0.50,0.50,0.50}

\draw[color=drawColor,line width= 0.6pt,line cap=round,line join=round,fill opacity=0.00,] ( 54.99, 83.14) -- ( 59.25, 83.14);

\draw[color=drawColor,line width= 0.6pt,line cap=round,line join=round,fill opacity=0.00,] ( 54.99,111.72) -- ( 59.25,111.72);

\draw[color=drawColor,line width= 0.6pt,line cap=round,line join=round,fill opacity=0.00,] ( 54.99,140.30) -- ( 59.25,140.30);

\draw[color=drawColor,line width= 0.6pt,line cap=round,line join=round,fill opacity=0.00,] ( 54.99,168.89) -- ( 59.25,168.89);

\draw[color=drawColor,line width= 0.6pt,line cap=round,line join=round,fill opacity=0.00,] ( 54.99,197.47) -- ( 59.25,197.47);
\end{scope}
\begin{scope}
\path[clip] (  0.00,  0.00) rectangle (289.08,433.62);
\end{scope}
\begin{scope}
\path[clip] (  0.00,  0.00) rectangle (289.08,433.62);
\end{scope}
\begin{scope}
\path[clip] (  0.00,  0.00) rectangle (289.08,433.62);
\end{scope}
\begin{scope}
\path[clip] ( 32.08, 55.41) rectangle ( 59.25, 55.41);
\end{scope}
\begin{scope}
\path[clip] (  0.00,  0.00) rectangle (289.08,433.62);
\end{scope}
\begin{scope}
\path[clip] ( 32.08, 32.08) rectangle ( 59.25, 55.41);
\end{scope}
\begin{scope}
\path[clip] (  0.00,  0.00) rectangle (289.08,433.62);
\end{scope}
\begin{scope}
\path[clip] ( 32.08, 32.08) rectangle ( 59.25, 32.08);
\end{scope}
\begin{scope}
\path[clip] (  0.00,  0.00) rectangle (289.08,433.62);
\end{scope}
\begin{scope}
\path[clip] ( 59.25,404.71) rectangle ( 59.25,404.71);
\end{scope}
\begin{scope}
\path[clip] (  0.00,  0.00) rectangle (289.08,433.62);
\end{scope}
\begin{scope}
\path[clip] ( 59.25,404.71) rectangle ( 59.25,404.71);
\end{scope}
\begin{scope}
\path[clip] (  0.00,  0.00) rectangle (289.08,433.62);
\end{scope}
\begin{scope}
\path[clip] ( 59.25,233.68) rectangle ( 59.25,404.71);
\end{scope}
\begin{scope}
\path[clip] (  0.00,  0.00) rectangle (289.08,433.62);
\end{scope}
\begin{scope}
\path[clip] ( 59.25,226.45) rectangle ( 59.25,233.68);
\end{scope}
\begin{scope}
\path[clip] (  0.00,  0.00) rectangle (289.08,433.62);
\end{scope}
\begin{scope}
\path[clip] ( 59.25, 55.41) rectangle ( 59.25,226.45);
\end{scope}
\begin{scope}
\path[clip] (  0.00,  0.00) rectangle (289.08,433.62);
\end{scope}
\begin{scope}
\path[clip] ( 59.25, 55.41) rectangle ( 59.25, 55.41);
\end{scope}
\begin{scope}
\path[clip] (  0.00,  0.00) rectangle (289.08,433.62);
\end{scope}
\begin{scope}
\path[clip] ( 59.25, 32.08) rectangle ( 59.25, 55.41);
\end{scope}
\begin{scope}
\path[clip] (  0.00,  0.00) rectangle (289.08,433.62);
\end{scope}
\begin{scope}
\path[clip] ( 59.25, 32.08) rectangle ( 59.25, 32.08);
\end{scope}
\begin{scope}
\path[clip] (  0.00,  0.00) rectangle (289.08,433.62);
\end{scope}
\begin{scope}
\path[clip] ( 59.25,404.71) rectangle (228.22,404.71);
\end{scope}
\begin{scope}
\path[clip] (  0.00,  0.00) rectangle (289.08,433.62);
\end{scope}
\begin{scope}
\path[clip] ( 59.25,404.71) rectangle (228.22,404.71);
\end{scope}
\begin{scope}
\path[clip] (  0.00,  0.00) rectangle (289.08,433.62);
\end{scope}
\begin{scope}
\path[clip] ( 59.25,233.68) rectangle (228.22,404.71);
\definecolor[named]{fillColor}{rgb}{0.90,0.90,0.90}

\draw[fill=fillColor,draw opacity=0.00,] ( 59.25,233.68) rectangle (228.22,404.71);
\definecolor[named]{drawColor}{rgb}{0.95,0.95,0.95}

\draw[color=drawColor,line width= 0.3pt,line cap=round,line join=round,fill opacity=0.00,] ( 59.25,238.63) --
	(228.22,238.63);

\draw[color=drawColor,line width= 0.3pt,line cap=round,line join=round,fill opacity=0.00,] ( 59.25,248.93) --
	(228.22,248.93);

\draw[color=drawColor,line width= 0.3pt,line cap=round,line join=round,fill opacity=0.00,] ( 59.25,259.22) --
	(228.22,259.22);

\draw[color=drawColor,line width= 0.3pt,line cap=round,line join=round,fill opacity=0.00,] ( 59.25,269.52) --
	(228.22,269.52);

\draw[color=drawColor,line width= 0.3pt,line cap=round,line join=round,fill opacity=0.00,] ( 59.25,279.82) --
	(228.22,279.82);

\draw[color=drawColor,line width= 0.3pt,line cap=round,line join=round,fill opacity=0.00,] ( 59.25,290.12) --
	(228.22,290.12);

\draw[color=drawColor,line width= 0.3pt,line cap=round,line join=round,fill opacity=0.00,] ( 59.25,300.42) --
	(228.22,300.42);

\draw[color=drawColor,line width= 0.3pt,line cap=round,line join=round,fill opacity=0.00,] ( 59.25,310.72) --
	(228.22,310.72);

\draw[color=drawColor,line width= 0.3pt,line cap=round,line join=round,fill opacity=0.00,] ( 59.25,321.02) --
	(228.22,321.02);

\draw[color=drawColor,line width= 0.3pt,line cap=round,line join=round,fill opacity=0.00,] ( 59.25,331.32) --
	(228.22,331.32);

\draw[color=drawColor,line width= 0.3pt,line cap=round,line join=round,fill opacity=0.00,] ( 59.25,341.61) --
	(228.22,341.61);

\draw[color=drawColor,line width= 0.3pt,line cap=round,line join=round,fill opacity=0.00,] ( 59.25,351.91) --
	(228.22,351.91);

\draw[color=drawColor,line width= 0.3pt,line cap=round,line join=round,fill opacity=0.00,] ( 59.25,362.21) --
	(228.22,362.21);

\draw[color=drawColor,line width= 0.3pt,line cap=round,line join=round,fill opacity=0.00,] ( 59.25,372.51) --
	(228.22,372.51);

\draw[color=drawColor,line width= 0.3pt,line cap=round,line join=round,fill opacity=0.00,] ( 59.25,382.81) --
	(228.22,382.81);

\draw[color=drawColor,line width= 0.3pt,line cap=round,line join=round,fill opacity=0.00,] ( 59.25,393.11) --
	(228.22,393.11);

\draw[color=drawColor,line width= 0.3pt,line cap=round,line join=round,fill opacity=0.00,] ( 59.25,403.41) --
	(228.22,403.41);

\draw[color=drawColor,line width= 0.3pt,line cap=round,line join=round,fill opacity=0.00,] ( 66.94,233.68) --
	( 66.94,404.71);

\draw[color=drawColor,line width= 0.3pt,line cap=round,line join=round,fill opacity=0.00,] ( 83.75,233.68) --
	( 83.75,404.71);

\draw[color=drawColor,line width= 0.3pt,line cap=round,line join=round,fill opacity=0.00,] (100.56,233.68) --
	(100.56,404.71);

\draw[color=drawColor,line width= 0.3pt,line cap=round,line join=round,fill opacity=0.00,] (117.38,233.68) --
	(117.38,404.71);

\draw[color=drawColor,line width= 0.3pt,line cap=round,line join=round,fill opacity=0.00,] (134.19,233.68) --
	(134.19,404.71);

\draw[color=drawColor,line width= 0.3pt,line cap=round,line join=round,fill opacity=0.00,] (151.00,233.68) --
	(151.00,404.71);

\draw[color=drawColor,line width= 0.3pt,line cap=round,line join=round,fill opacity=0.00,] (167.82,233.68) --
	(167.82,404.71);

\draw[color=drawColor,line width= 0.3pt,line cap=round,line join=round,fill opacity=0.00,] (184.63,233.68) --
	(184.63,404.71);

\draw[color=drawColor,line width= 0.3pt,line cap=round,line join=round,fill opacity=0.00,] (201.44,233.68) --
	(201.44,404.71);

\draw[color=drawColor,line width= 0.3pt,line cap=round,line join=round,fill opacity=0.00,] (218.26,233.68) --
	(218.26,404.71);
\definecolor[named]{drawColor}{rgb}{1.00,1.00,1.00}

\draw[color=drawColor,line width= 0.6pt,line cap=round,line join=round,fill opacity=0.00,] ( 59.25,248.93) --
	(228.22,248.93);

\draw[color=drawColor,line width= 0.6pt,line cap=round,line join=round,fill opacity=0.00,] ( 59.25,269.52) --
	(228.22,269.52);

\draw[color=drawColor,line width= 0.6pt,line cap=round,line join=round,fill opacity=0.00,] ( 59.25,290.12) --
	(228.22,290.12);

\draw[color=drawColor,line width= 0.6pt,line cap=round,line join=round,fill opacity=0.00,] ( 59.25,310.72) --
	(228.22,310.72);

\draw[color=drawColor,line width= 0.6pt,line cap=round,line join=round,fill opacity=0.00,] ( 59.25,331.32) --
	(228.22,331.32);

\draw[color=drawColor,line width= 0.6pt,line cap=round,line join=round,fill opacity=0.00,] ( 59.25,351.91) --
	(228.22,351.91);

\draw[color=drawColor,line width= 0.6pt,line cap=round,line join=round,fill opacity=0.00,] ( 59.25,372.51) --
	(228.22,372.51);

\draw[color=drawColor,line width= 0.6pt,line cap=round,line join=round,fill opacity=0.00,] ( 59.25,393.11) --
	(228.22,393.11);

\draw[color=drawColor,line width= 0.6pt,line cap=round,line join=round,fill opacity=0.00,] ( 66.94,233.68) --
	( 66.94,404.71);

\draw[color=drawColor,line width= 0.6pt,line cap=round,line join=round,fill opacity=0.00,] (100.56,233.68) --
	(100.56,404.71);

\draw[color=drawColor,line width= 0.6pt,line cap=round,line join=round,fill opacity=0.00,] (134.19,233.68) --
	(134.19,404.71);

\draw[color=drawColor,line width= 0.6pt,line cap=round,line join=round,fill opacity=0.00,] (167.82,233.68) --
	(167.82,404.71);

\draw[color=drawColor,line width= 0.6pt,line cap=round,line join=round,fill opacity=0.00,] (201.44,233.68) --
	(201.44,404.71);
\definecolor[named]{drawColor}{rgb}{0.00,0.00,0.00}

\draw[color=drawColor,line width= 0.6pt,line join=round,fill opacity=0.00,] ( 66.94,276.98) --
	( 76.36,287.08) --
	( 77.78,287.95) --
	(112.12,303.73) --
	(115.80,305.98) --
	(118.64,306.85) --
	(130.94,308.73);

\draw[color=drawColor,line width= 0.6pt,line join=round,fill opacity=0.00,] ( 66.94,241.45) --
	(112.12,303.73) --
	(115.80,305.98) --
	(118.64,306.85) --
	(130.94,308.73);

\draw[color=drawColor,line width= 0.6pt,line join=round,fill opacity=0.00,] ( 66.94,281.86) --
	( 67.12,281.94) --
	( 72.94,285.51) --
	( 76.36,287.08) --
	( 77.78,287.95) --
	(112.12,303.73) --
	(115.80,305.98) --
	(118.64,306.85) --
	(130.94,308.73);

\draw[color=drawColor,line width= 0.6pt,line join=round,fill opacity=0.00,] ( 66.94,289.61) --
	( 77.78,287.95) --
	(112.12,303.73) --
	(115.80,305.98) --
	(118.64,306.85) --
	(130.94,308.73);

\draw[color=drawColor,line width= 0.6pt,line join=round,fill opacity=0.00,] ( 66.94,284.59) --
	( 72.94,285.51) --
	( 76.36,287.08) --
	( 77.78,287.95) --
	(112.12,303.73) --
	(115.80,305.98) --
	(118.64,306.85) --
	(130.94,308.73);

\draw[color=drawColor,line width= 0.6pt,line join=round,fill opacity=0.00,] ( 66.94,340.79) --
	( 98.80,316.40) --
	(115.80,305.98) --
	(118.64,306.85) --
	(130.94,308.73);

\draw[color=drawColor,line width= 0.6pt,line join=round,fill opacity=0.00,] ( 66.94,396.94) --
	(130.94,308.73);

\draw[color=drawColor,line width= 0.6pt,line join=round,fill opacity=0.00,] ( 66.94,362.27) --
	(118.64,306.85) --
	(130.94,308.73);

\draw[color=drawColor,line width= 0.6pt,line join=round,fill opacity=0.00,] ( 66.94,331.04) --
	( 98.80,316.40) --
	(115.80,305.98) --
	(118.64,306.85) --
	(130.94,308.73);

\draw[color=drawColor,line width= 0.6pt,line join=round,fill opacity=0.00,] ( 66.94,281.80) --
	( 67.12,281.94) --
	( 72.94,285.51) --
	( 76.36,287.08) --
	( 77.78,287.95) --
	(112.12,303.73) --
	(115.80,305.98) --
	(118.64,306.85) --
	(130.94,308.73);
\definecolor[named]{fillColor}{rgb}{0.00,0.00,0.00}

\draw[fill=fillColor,draw opacity=0.00,] ( 66.94,281.80) circle (  2.13);
\definecolor[named]{drawColor}{rgb}{1.00,0.00,0.00}

\draw[color=drawColor,line width= 0.6pt,line join=round,fill opacity=0.00,] (187.99,233.68) -- (187.99,404.71);
\end{scope}
\begin{scope}
\path[clip] (  0.00,  0.00) rectangle (289.08,433.62);
\end{scope}
\begin{scope}
\path[clip] ( 59.25,226.45) rectangle (228.22,233.68);
\end{scope}
\begin{scope}
\path[clip] (  0.00,  0.00) rectangle (289.08,433.62);
\end{scope}
\begin{scope}
\path[clip] ( 59.25, 55.41) rectangle (228.22,226.45);
\definecolor[named]{fillColor}{rgb}{0.90,0.90,0.90}

\draw[fill=fillColor,draw opacity=0.00,] ( 59.25, 55.41) rectangle (228.22,226.45);
\definecolor[named]{drawColor}{rgb}{0.95,0.95,0.95}

\draw[color=drawColor,line width= 0.3pt,line cap=round,line join=round,fill opacity=0.00,] ( 59.25, 68.85) --
	(228.22, 68.85);

\draw[color=drawColor,line width= 0.3pt,line cap=round,line join=round,fill opacity=0.00,] ( 59.25, 83.14) --
	(228.22, 83.14);

\draw[color=drawColor,line width= 0.3pt,line cap=round,line join=round,fill opacity=0.00,] ( 59.25, 97.43) --
	(228.22, 97.43);

\draw[color=drawColor,line width= 0.3pt,line cap=round,line join=round,fill opacity=0.00,] ( 59.25,111.72) --
	(228.22,111.72);

\draw[color=drawColor,line width= 0.3pt,line cap=round,line join=round,fill opacity=0.00,] ( 59.25,126.01) --
	(228.22,126.01);

\draw[color=drawColor,line width= 0.3pt,line cap=round,line join=round,fill opacity=0.00,] ( 59.25,140.30) --
	(228.22,140.30);

\draw[color=drawColor,line width= 0.3pt,line cap=round,line join=round,fill opacity=0.00,] ( 59.25,154.60) --
	(228.22,154.60);

\draw[color=drawColor,line width= 0.3pt,line cap=round,line join=round,fill opacity=0.00,] ( 59.25,168.89) --
	(228.22,168.89);

\draw[color=drawColor,line width= 0.3pt,line cap=round,line join=round,fill opacity=0.00,] ( 59.25,183.18) --
	(228.22,183.18);

\draw[color=drawColor,line width= 0.3pt,line cap=round,line join=round,fill opacity=0.00,] ( 59.25,197.47) --
	(228.22,197.47);

\draw[color=drawColor,line width= 0.3pt,line cap=round,line join=round,fill opacity=0.00,] ( 59.25,211.76) --
	(228.22,211.76);

\draw[color=drawColor,line width= 0.3pt,line cap=round,line join=round,fill opacity=0.00,] ( 59.25,226.05) --
	(228.22,226.05);

\draw[color=drawColor,line width= 0.3pt,line cap=round,line join=round,fill opacity=0.00,] ( 66.94, 55.41) --
	( 66.94,226.45);

\draw[color=drawColor,line width= 0.3pt,line cap=round,line join=round,fill opacity=0.00,] ( 83.75, 55.41) --
	( 83.75,226.45);

\draw[color=drawColor,line width= 0.3pt,line cap=round,line join=round,fill opacity=0.00,] (100.56, 55.41) --
	(100.56,226.45);

\draw[color=drawColor,line width= 0.3pt,line cap=round,line join=round,fill opacity=0.00,] (117.38, 55.41) --
	(117.38,226.45);

\draw[color=drawColor,line width= 0.3pt,line cap=round,line join=round,fill opacity=0.00,] (134.19, 55.41) --
	(134.19,226.45);

\draw[color=drawColor,line width= 0.3pt,line cap=round,line join=round,fill opacity=0.00,] (151.00, 55.41) --
	(151.00,226.45);

\draw[color=drawColor,line width= 0.3pt,line cap=round,line join=round,fill opacity=0.00,] (167.82, 55.41) --
	(167.82,226.45);

\draw[color=drawColor,line width= 0.3pt,line cap=round,line join=round,fill opacity=0.00,] (184.63, 55.41) --
	(184.63,226.45);

\draw[color=drawColor,line width= 0.3pt,line cap=round,line join=round,fill opacity=0.00,] (201.44, 55.41) --
	(201.44,226.45);

\draw[color=drawColor,line width= 0.3pt,line cap=round,line join=round,fill opacity=0.00,] (218.26, 55.41) --
	(218.26,226.45);
\definecolor[named]{drawColor}{rgb}{1.00,1.00,1.00}

\draw[color=drawColor,line width= 0.6pt,line cap=round,line join=round,fill opacity=0.00,] ( 59.25, 83.14) --
	(228.22, 83.14);

\draw[color=drawColor,line width= 0.6pt,line cap=round,line join=round,fill opacity=0.00,] ( 59.25,111.72) --
	(228.22,111.72);

\draw[color=drawColor,line width= 0.6pt,line cap=round,line join=round,fill opacity=0.00,] ( 59.25,140.30) --
	(228.22,140.30);

\draw[color=drawColor,line width= 0.6pt,line cap=round,line join=round,fill opacity=0.00,] ( 59.25,168.89) --
	(228.22,168.89);

\draw[color=drawColor,line width= 0.6pt,line cap=round,line join=round,fill opacity=0.00,] ( 59.25,197.47) --
	(228.22,197.47);

\draw[color=drawColor,line width= 0.6pt,line cap=round,line join=round,fill opacity=0.00,] ( 66.94, 55.41) --
	( 66.94,226.45);

\draw[color=drawColor,line width= 0.6pt,line cap=round,line join=round,fill opacity=0.00,] (100.56, 55.41) --
	(100.56,226.45);

\draw[color=drawColor,line width= 0.6pt,line cap=round,line join=round,fill opacity=0.00,] (134.19, 55.41) --
	(134.19,226.45);

\draw[color=drawColor,line width= 0.6pt,line cap=round,line join=round,fill opacity=0.00,] (167.82, 55.41) --
	(167.82,226.45);

\draw[color=drawColor,line width= 0.6pt,line cap=round,line join=round,fill opacity=0.00,] (201.44, 55.41) --
	(201.44,226.45);
\definecolor[named]{drawColor}{rgb}{0.00,0.00,0.00}

\draw[color=drawColor,line width= 0.6pt,line join=round,fill opacity=0.00,] ( 66.94, 83.25) --
	( 83.24, 84.63) --
	(107.04, 90.70) --
	(220.54,138.94);

\draw[color=drawColor,line width= 0.6pt,line join=round,fill opacity=0.00,] ( 66.94, 63.19) --
	( 69.83, 65.40) --
	(107.04, 90.70) --
	(220.54,138.94);

\draw[color=drawColor,line width= 0.6pt,line join=round,fill opacity=0.00,] ( 66.94, 63.68) --
	( 69.83, 65.40) --
	(107.04, 90.70) --
	(220.54,138.94);

\draw[color=drawColor,line width= 0.6pt,line join=round,fill opacity=0.00,] ( 66.94, 80.14) --
	( 79.30, 83.30) --
	( 83.24, 84.63) --
	(107.04, 90.70) --
	(220.54,138.94);

\draw[color=drawColor,line width= 0.6pt,line join=round,fill opacity=0.00,] ( 66.94, 78.04) --
	( 79.30, 83.30) --
	( 83.24, 84.63) --
	(107.04, 90.70) --
	(220.54,138.94);

\draw[color=drawColor,line width= 0.6pt,line join=round,fill opacity=0.00,] ( 66.94,213.10) --
	( 84.20,205.76) --
	(101.76,195.31) --
	(171.11,159.94) --
	(220.54,138.94);

\draw[color=drawColor,line width= 0.6pt,line join=round,fill opacity=0.00,] ( 66.94,218.67) --
	( 80.74,208.11) --
	( 84.20,205.76) --
	(101.76,195.31) --
	(171.11,159.94) --
	(220.54,138.94);

\draw[color=drawColor,line width= 0.6pt,line join=round,fill opacity=0.00,] ( 66.94,216.33) --
	( 80.74,208.11) --
	( 84.20,205.76) --
	(101.76,195.31) --
	(171.11,159.94) --
	(220.54,138.94);

\draw[color=drawColor,line width= 0.6pt,line join=round,fill opacity=0.00,] ( 66.94,204.19) --
	(101.76,195.31) --
	(171.11,159.94) --
	(220.54,138.94);

\draw[color=drawColor,line width= 0.6pt,line join=round,fill opacity=0.00,] ( 66.94,168.80) --
	(171.11,159.94) --
	(220.54,138.94);
\definecolor[named]{fillColor}{rgb}{0.00,0.00,0.00}

\draw[fill=fillColor,draw opacity=0.00,] ( 66.94,168.80) circle (  2.13);
\definecolor[named]{drawColor}{rgb}{1.00,0.00,0.00}

\draw[color=drawColor,line width= 0.6pt,line join=round,fill opacity=0.00,] (187.99, 55.41) -- (187.99,226.45);
\end{scope}
\begin{scope}
\path[clip] (  0.00,  0.00) rectangle (289.08,433.62);
\end{scope}
\begin{scope}
\path[clip] ( 59.25, 55.41) rectangle (228.22, 55.41);
\end{scope}
\begin{scope}
\path[clip] (  0.00,  0.00) rectangle (289.08,433.62);
\end{scope}
\begin{scope}
\path[clip] (  0.00,  0.00) rectangle (289.08,433.62);
\definecolor[named]{drawColor}{rgb}{0.00,0.00,0.00}

\node[color=drawColor,anchor=base,inner sep=0pt, outer sep=0pt, scale=  0.42] at ( 66.94, 32.08) {0.00%
};

\node[color=drawColor,anchor=base,inner sep=0pt, outer sep=0pt, scale=  0.42] at (100.56, 32.08) {0.05%
};

\node[color=drawColor,anchor=base,inner sep=0pt, outer sep=0pt, scale=  0.42] at (134.19, 32.08) {0.10%
};

\node[color=drawColor,anchor=base,inner sep=0pt, outer sep=0pt, scale=  0.42] at (167.82, 32.08) {0.15%
};

\node[color=drawColor,anchor=base,inner sep=0pt, outer sep=0pt, scale=  0.42] at (201.44, 32.08) {0.20%
};
\end{scope}
\begin{scope}
\path[clip] (  0.00,  0.00) rectangle (289.08,433.62);
\definecolor[named]{drawColor}{rgb}{0.50,0.50,0.50}

\draw[color=drawColor,line width= 0.6pt,line cap=round,line join=round,fill opacity=0.00,] ( 66.94, 51.14) -- ( 66.94, 55.41);

\draw[color=drawColor,line width= 0.6pt,line cap=round,line join=round,fill opacity=0.00,] (100.56, 51.14) -- (100.56, 55.41);

\draw[color=drawColor,line width= 0.6pt,line cap=round,line join=round,fill opacity=0.00,] (134.19, 51.14) -- (134.19, 55.41);

\draw[color=drawColor,line width= 0.6pt,line cap=round,line join=round,fill opacity=0.00,] (167.82, 51.14) -- (167.82, 55.41);

\draw[color=drawColor,line width= 0.6pt,line cap=round,line join=round,fill opacity=0.00,] (201.44, 51.14) -- (201.44, 55.41);
\end{scope}
\begin{scope}
\path[clip] (  0.00,  0.00) rectangle (289.08,433.62);
\end{scope}
\begin{scope}
\path[clip] (  0.00,  0.00) rectangle (289.08,433.62);
\end{scope}
\begin{scope}
\path[clip] (  0.00,  0.00) rectangle (289.08,433.62);
\end{scope}
\begin{scope}
\path[clip] ( 59.25, 32.08) rectangle (228.22, 32.08);
\end{scope}
\begin{scope}
\path[clip] (  0.00,  0.00) rectangle (289.08,433.62);
\end{scope}
\begin{scope}
\path[clip] (228.22,404.71) rectangle (228.22,404.71);
\end{scope}
\begin{scope}
\path[clip] (  0.00,  0.00) rectangle (289.08,433.62);
\end{scope}
\begin{scope}
\path[clip] (228.22,404.71) rectangle (228.22,404.71);
\end{scope}
\begin{scope}
\path[clip] (  0.00,  0.00) rectangle (289.08,433.62);
\end{scope}
\begin{scope}
\path[clip] (228.22,233.68) rectangle (228.22,404.71);
\end{scope}
\begin{scope}
\path[clip] (  0.00,  0.00) rectangle (289.08,433.62);
\end{scope}
\begin{scope}
\path[clip] (228.22,226.45) rectangle (228.22,233.68);
\end{scope}
\begin{scope}
\path[clip] (  0.00,  0.00) rectangle (289.08,433.62);
\end{scope}
\begin{scope}
\path[clip] (228.22, 55.41) rectangle (228.22,226.45);
\end{scope}
\begin{scope}
\path[clip] (  0.00,  0.00) rectangle (289.08,433.62);
\end{scope}
\begin{scope}
\path[clip] (228.22, 55.41) rectangle (228.22, 55.41);
\end{scope}
\begin{scope}
\path[clip] (  0.00,  0.00) rectangle (289.08,433.62);
\end{scope}
\begin{scope}
\path[clip] (228.22, 32.08) rectangle (228.22, 55.41);
\end{scope}
\begin{scope}
\path[clip] (  0.00,  0.00) rectangle (289.08,433.62);
\end{scope}
\begin{scope}
\path[clip] (228.22, 32.08) rectangle (228.22, 32.08);
\end{scope}
\begin{scope}
\path[clip] (  0.00,  0.00) rectangle (289.08,433.62);
\end{scope}
\begin{scope}
\path[clip] (228.22,404.71) rectangle (245.72,404.71);
\end{scope}
\begin{scope}
\path[clip] (  0.00,  0.00) rectangle (289.08,433.62);
\end{scope}
\begin{scope}
\path[clip] (228.22,404.71) rectangle (245.72,404.71);
\end{scope}
\begin{scope}
\path[clip] (  0.00,  0.00) rectangle (289.08,433.62);
\end{scope}
\begin{scope}
\path[clip] (228.22,233.68) rectangle (245.72,404.71);
\definecolor[named]{fillColor}{rgb}{0.80,0.80,0.80}

\draw[fill=fillColor,draw opacity=0.00,] (228.22,233.68) rectangle (245.72,404.71);
\definecolor[named]{drawColor}{rgb}{0.00,0.00,0.00}

\node[rotate=270.00,color=drawColor,anchor=base,inner sep=0pt, outer sep=0pt, scale=  0.40] at (235.45,319.19) {$\alpha^1$%
};
\end{scope}
\begin{scope}
\path[clip] (  0.00,  0.00) rectangle (289.08,433.62);
\end{scope}
\begin{scope}
\path[clip] (228.22,226.45) rectangle (245.72,233.68);
\end{scope}
\begin{scope}
\path[clip] (  0.00,  0.00) rectangle (289.08,433.62);
\end{scope}
\begin{scope}
\path[clip] (228.22, 55.41) rectangle (245.72,226.45);
\definecolor[named]{fillColor}{rgb}{0.80,0.80,0.80}

\draw[fill=fillColor,draw opacity=0.00,] (228.22, 55.41) rectangle (245.72,226.45);
\definecolor[named]{drawColor}{rgb}{0.00,0.00,0.00}

\node[rotate=270.00,color=drawColor,anchor=base,inner sep=0pt, outer sep=0pt, scale=  0.40] at (235.45,140.93) {$\alpha^2$%
};
\end{scope}
\begin{scope}
\path[clip] (  0.00,  0.00) rectangle (289.08,433.62);
\end{scope}
\begin{scope}
\path[clip] (228.22, 55.41) rectangle (245.72, 55.41);
\end{scope}
\begin{scope}
\path[clip] (  0.00,  0.00) rectangle (289.08,433.62);
\end{scope}
\begin{scope}
\path[clip] (228.22, 32.08) rectangle (245.72, 55.41);
\end{scope}
\begin{scope}
\path[clip] (  0.00,  0.00) rectangle (289.08,433.62);
\end{scope}
\begin{scope}
\path[clip] (228.22, 32.08) rectangle (245.72, 32.08);
\end{scope}
\begin{scope}
\path[clip] (  0.00,  0.00) rectangle (289.08,433.62);
\end{scope}
\begin{scope}
\path[clip] (245.72,404.71) rectangle (245.72,404.71);
\end{scope}
\begin{scope}
\path[clip] (  0.00,  0.00) rectangle (289.08,433.62);
\end{scope}
\begin{scope}
\path[clip] (245.72,404.71) rectangle (245.72,404.71);
\end{scope}
\begin{scope}
\path[clip] (  0.00,  0.00) rectangle (289.08,433.62);
\end{scope}
\begin{scope}
\path[clip] (245.72,233.68) rectangle (245.72,404.71);
\end{scope}
\begin{scope}
\path[clip] (  0.00,  0.00) rectangle (289.08,433.62);
\end{scope}
\begin{scope}
\path[clip] (245.72,226.45) rectangle (245.72,233.68);
\end{scope}
\begin{scope}
\path[clip] (  0.00,  0.00) rectangle (289.08,433.62);
\end{scope}
\begin{scope}
\path[clip] (245.72, 55.41) rectangle (245.72,226.45);
\end{scope}
\begin{scope}
\path[clip] (  0.00,  0.00) rectangle (289.08,433.62);
\end{scope}
\begin{scope}
\path[clip] (245.72, 55.41) rectangle (245.72, 55.41);
\end{scope}
\begin{scope}
\path[clip] (  0.00,  0.00) rectangle (289.08,433.62);
\end{scope}
\begin{scope}
\path[clip] (245.72, 32.08) rectangle (245.72, 55.41);
\end{scope}
\begin{scope}
\path[clip] (  0.00,  0.00) rectangle (289.08,433.62);
\end{scope}
\begin{scope}
\path[clip] (245.72, 32.08) rectangle (245.72, 32.08);
\end{scope}
\begin{scope}
\path[clip] (  0.00,  0.00) rectangle (289.08,433.62);
\end{scope}
\begin{scope}
\path[clip] (  0.00,  0.00) rectangle (289.08,433.62);
\end{scope}
\begin{scope}
\path[clip] (  0.00,  0.00) rectangle (289.08,433.62);
\end{scope}
\end{tikzpicture}


\begin{block}{Experiments}
{\bfseries Background subtraction}~{\small (Cehver '08, Huang '09, Wright '08)}
\begin{itemize}
\item $\y = \X\w + \e$; $\X$, background images; $\e$ is sparse and structured.
\item $p \approx 60\,000$; $|\GG| \approx 120\,000$; $|E| \approx 660\,000$; $1.5sec / \text{prox}$
\end{itemize}
   \begin{figure}
      \centering
%      \includegraphics[width=6.6cm]{images/original_trees.png} \hfill
%      \includegraphics[width=6.6cm]{images/background_trees_struct.png}\hfill
%      \includegraphics[width=6.6cm]{images/foreground_trees_L1.png}\hfill 
%      \includegraphics[width=6.6cm]{images/foreground_trees_struct.png}\hfill 
%      \includegraphics[width=6.6cm]{images/foreground_trees_struct_BIS.png} \\
%     \vspace*{0.2cm}
%       \includegraphics[width=6.6cm]{images/original_video_boot.png}\hfill
%       \includegraphics[width=6.6cm]{images/background_video_boot_struct.png}\hfill
%       \includegraphics[width=6.6cm]{images/foreground_video_boot_L1.png}\hfill
%       \includegraphics[width=6.6cm]{images/foreground_video_boot_struct.png}\hfill 
%       \includegraphics[width=6.6cm]{images/foreground_video_boot_struct_BIS.png} \\
    \caption{{\small Original, est. background, est. foreground with $\ell_1$, with $\ell_1+\Omega$.}}
   \end{figure}
\vspace*{-0.3cm}
{\bfseries Multi-task learning of hierarchical structures}
   \begin{displaymath}
    \min_{\X,\W}
    \frac{1}{n}\sum_{i=1}^n\!\Big[\frac{1}{2} \|\y^i-\X\w^i\|_2^2 + \lambda_1 \Omega_{\text{tree}}(\w^i)\Big]\!+\!\lambda_2\Omega_{\text{joint}}(\W),\ \text{s.t.}\
    \forall j,~~\|\x^j\|_2\leq 1, 
\end{displaymath}
$n=10\,000$ image patches; $p,|\GG| \approx 4\,000\,000$; $|E|\approx 12\,000\,000$
\begin{figure}[hbtp]
   \centering
%   \includegraphics[width=0.42\linewidth]{images/tree2.png}\hfill
%   \includegraphics[width=0.54\linewidth]{images/tree_denois.png} 
   \caption{Example of hierarchy - Mean square error versus dictionary size. 
   } \label{fig:tree}
\end{figure}
\end{block}
\end{column}
\end{columns}

\end{frame}
\end{document}

